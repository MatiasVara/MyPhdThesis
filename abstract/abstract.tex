\addcontentsline{toc}{chapter}{R\'esum\'e}

\begin{frenchabstract}
Les appareils modernes sont constitu\'es des plusieurs sous-syst\`emes de diff\'erentes sortes qui doivent communiquer et interagir. L'h\'et\'erog\'en\'eit\'e de ces sous-syst\`eme et leurs interactions complexes rendent tr\`es d\'elicate leur d\'eveloppement. L'approche d'ing\'enierie dirig\'ee par les mod\`eles apporte une solution en permettant l'expression de nombreux mod\'eles structurels et comportementaux de natures tr\`es diverses. Dans ce contexte, il est n\'ecessaire de construire un mod\`ele unique qui int\`egre ces diff\'erents mod\`eles afin d'y appliquer des m\'ethodes de validation et de v\'erification ce qui permettent aux ing\'enieurs syst\`eme de comprendre et de valider un comportement global. Cependant, la coordination manuelle des diff\'erents mod\`eles qui composent le syst\`eme est une op\'eration source d'erreurs et les approches automatiques ne concernent qu'un ensemble possibles des patrons de coordination. Dans ces approches, le patron de coordination est souvent encapsul\'e dans un outil dont il est difficile d'extraire les liens avec le syst\`eme global. Cette th\`ese propose le Behavioral Coordination Operator Language (\bcool), un langage d\'edi\'e \`a sp\'ecification des patrons de coordination entre des langages \`a partir de la d\'efinition des op\'erateurs. Ces op\'erateurs sont alors employ\'es afin d'automatiser la coordination de mod\`eles exprim\'es dans ces langages. \bcool est impl\'ement\'e comme une suite de plugins pour Eclipse Modeling Framework et pr\'esente ainsi un environnement complet pour l'ex\'ecution et la v\'erification de diff\'erents mod\`eles coordonn\'es. Nous illustrons cette approche avec la d\'efinition d'op\'erateurs de coordination entre finite state machines et activity diagrams. Ensuite, nous utilisons ces op\'erateurs afin de coordonner et d'ex\'ecuter un mod\`ele h\'et\'erog\`ene de cam\'era de surveillance.
\end{frenchabstract}



\cleardoublepage
\addcontentsline{toc}{chapter}{Abstract}


\begin{abstract}
Modern devices embed several subsystems with different characteristics that communicate and interact in many ways. This makes its development complex since a designer has to deal with the heterogeneity of each subsystem but also with the interaction between them. To tackle the development of complex systems, Model Driven Engineering promotes the use of various, possibly heterogeneous, structural and behavioral models. In this context, the coordination of behavioral models to produce a single integrated model is necessary to provide support for validation and verification. It allows system designers to understand and validate the global and emerging behavior of the system. However, the manual coordination of models is tedious and error-prone, and current approaches to automate the coordination are bound to a fixed set of coordination patterns. Moreover, they encode the pattern into a tool thus limiting reasoning on the global system behavior. In this thesis, we propose a Behavioral Coordination Operator Language (\bcool) to reify coordination patterns between specific domains by using coordination operators between the Domain-Specific Modeling Languages used in these domains. Those operators are then used to automate the coordination of models conforming to these languages. \bcool is implemented as plugins for the Eclipse Modeling Framework thus providing a complete environment to execute and verify coordinated models. We illustrate the use of \bcool with the definition of coordination operators between timed finite state machines and activity diagrams. We then use these operators to coordinate and execute the heterogeneous model of a surveillance camera system.
\end{abstract}