\addcontentsline{toc}{chapter}{R\'esum\'e}

\begin{frenchabstract}
Les appareils modernes sont constitués de plusieurs sous-systèmes de différentes sortes qui communiquent et interagissent. L'hétérogénéité de ces sous-systèmes et leurs interactions complexes rendent très délicate leur développement. L'approche d'ingénierie dirigée par les modèles apporte une solution en permettant l'expression de nombreux modèles structurels et comportementaux de natures très diverses. Dans ce contexte, il est nécessaire de construire un modèle unique qui intègre ces différents modèles afin d'y appliquer des méthodes de validation et de vérification pour permettre aux ingénieurs système de comprendre et de valider un comportement global. Cependant, la coordination manuelle des différents modèles qui composent le système est une opération source d'erreurs et les approches automatiques proposent des patrons de coordination ad-hoc pour certaines paires de langages. Dans ces approches, le patron de coordination est souvent encapsulé dans un outil dont il est difficile d'extraire les liens avec le système global. Cette thèse propose le Behavioral Coordination Operator Language (\bcool), un langage dédié à la spécification de patrons de coordination entre des langages à partir de la définition d'opérateurs de coordination. Ces opérateurs sont employés afin d'automatiser la coordination de modèles exprimés dans ces langages. \bcool est implémenté comme une suite de plugins qui s'appuient sur l'Eclipse Modeling Framework et présente ainsi un environnement complet pour l'exécution et la vérification de différents modèles coordonnés. Nous illustrons cette approche avec la définition d'opérateurs de coordination entre deux langages: \emph{timed finite state machines} et  \emph{activities}. Ensuite, nous utilisons ces opérateurs afin de coordonner et d'exécuter un modèle hétérogène de caméra de surveillance.
\end{frenchabstract}



\cleardoublepage
\addcontentsline{toc}{chapter}{Abstract}


\begin{abstract}
Modern devices embed several subsystems with different characteristics that communicate and interact in many ways. This makes its development complex since a designer has to deal with the heterogeneity of each subsystem but also with the interactions among them. To tackle the development of complex systems, Model Driven Engineering promotes the use of various, possibly heterogeneous, structural and behavioral models. In this context, the coordination of behavioral models to produce a single integrated model is necessary to provide support for validation and verification. It allows system designers to understand and validate the global and emerging behavior of the system. However, the manual coordination of models is tedious and error-prone, and current approaches to automate the coordination are bound to a fixed set of coordination patterns. Moreover, automatic approaches encode the pattern into a tool thus limiting the reasoning on the global system behavior. In this thesis, we propose the Behavioral Coordination Operator Language (\bcool) to reify coordination patterns between specific domains by using coordination operators between the Domain-Specific Modeling Languages used in these domains. Those operators are then used to automate the coordination of models conforming to these languages. \bcool is implemented as plugins for the Eclipse Modeling Framework thus providing a complete environment to execute and verify coordinated models. We illustrate the use of \bcool with the definition of coordination operators between two languages: timed finite state machines and activities. We then use these operators to coordinate and execute the heterogeneous model of a surveillance camera system.
\end{abstract}