\chapter{Introduction}

%\section{Context}

\begin{itemize}
	% context 
	% complex systems
	\item Nowadays devices are becoming more complex. They embed several subsystems with different characteristics that communicate and interact in many ways. For example, current cars can integrate an adaptive control cruise system, GPS tracking, fuel control system and soon on. Furthermore, these subsystems are widely coupled. For instance, the adaptive cruise system controls the way to get home depending on the GPS tacking, also, the fuel control system regulates the speed of the car depending on the level of fuel.
	
	% designing of complex systems
	\item This makes the designing of these systems very complex. A designer has to dealt with the heterogeneity of each subsystem but also with the interaction between them. To reduce the complexity of such a development, the designing is split into different domains, \eg mechanical, electronic, software. Therefore, the development of such complex system is talcked by different domains experts.
	
	% Multiple DSMLs --> heterogenenous
	\item In this context, Model Driven Engineering (MDE) has addressed the problem of the development of application for complex systems by proposing Domain Specific Modeling Languages (DSMLs). Each domain expert relies on a DSMLs to better describe its domain. Each DSMLs has its own expressiveness and properties. As a result of such development, several models conforming to different DSMLs are developed and the specification of the overall system becomes \emph{heterogeneous}.
% problem
	\item However, at some point of development, these models have to be integrated to understand the system and its emerging behavior globally. It is necessary to specify how models and languages are related to each other, in both a structural and a behavioral way. Whereas the MDE community provides some extensive support for the structural composition of models and languages (\eg~\cite{kompose,epsilon}), in this thesis, we rather focus on the coordination~\cite{coordsignibib} of behavioral models/languages to provide simulation and/or verification capabilities for the whole system specification. 
	
% partial solution
	\item Coordination Languages~\cite{coordsignibib} and Architecture Description Languages (ADLs)~\cite{frameadlsbib} provide dedicated languages to specify the coordination between particular behavioral models. This is usually done by system designers that apply some coordination patterns according to their own skills and know-how. To automate such a task, coordination frameworks~\cite{ptoleframebib,modhelxbib} have identified such a pattern and they have encoded it inside a tool, \eg Ptolemy~\cite{ptoleframebib}. However, the semantics of the coordination is hidden thus limiting reasoning. Furthermore, they rely on a general purpose language to express the coordination thus limiting verification and validation of the coordinated system. Since the coordination pattern is encoded, they prevent any tunning of the pattern.   
	
% contribution
\item In this thesis, we deal with the coordination of heterogeneous behavioral models by leveraging on the system designer's skills. We propose a dedicated language named \bcool (standing as Behavioral Coordination Operator Language) that allows for capturing coordination patterns for a given set of DSMLs. These patterns are captured between languages, and then used to derive a coordination model automatically for models conforming to the targeted DSMLs. The coordination at the language level relies on a so-called \emph{language behavioral interface}. This interface exposes an abstraction of the language behavioral semantics in terms of Events. \bcool helps understand and reason about the relationships between different languages used in the design of a system. 

% outline	 		
\item The content of this thesis is organized as follows. 

\item Chapter~\ref{ch:background} presents the thesis background.

\item Chapter~\ref{ch:framework} presents a framework 


%\item Section~\ref{sec:coord-lang} presents the main issues in the coordination of behavioral models, and shows how they can be tackled by explicitly capturing coordination patterns at the language level. Section~\ref{sec:interfaceandexample} defines the notion of language behavioral interface by using an example language named Timed Finite State Machine (TFSM). This language is used later in Section~\ref{sec:BCOol} to illustrate \bcool. In Section~\ref{sec:caseStudies}, we validate the approach by using \bcool to capture three coordination patterns between two languages: TFSM and fUML Activities. Section~\ref{sec:related} gives an overview and comparison to related work. Section~\ref{sec:conclu} concludes with a brief summary and a discussion of ongoing and future actions.
	
%\item Finally, we provide the conclusion of this work, highlighting its main contributions and we give some future perspectives in Chapter 7.
	
\end{itemize}

\begin{landscape}
\begin{figure}
	\begin{center}
		\includegraphics[width=1\textwidth]{Thesisoutline.pdf}
		\label{fig:thesis outline}
	\end{center}
\end{figure}
\end{landscape}

%\section{Problem}

%\section{Contributions}

%Contributions here.

%\section{Publications}

%Publications here.