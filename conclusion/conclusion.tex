\chapter{Conclusion}
\label{ch:conclusions}
In this thesis, we have managed the coordination of heterogeneous behavioral models by specifying coordination patterns at language level. To capture the specification of these patters, we have proposed \bcool, a dedicated language for integrators. This specification at language level enables a system designer to automate the coordination between particular models. In the following, we summarize the main findings of this thesis. We finish this chapter by proposing future work.    

\section{Overview}
	
	
This thesis has focused on the coordination of heterogeneous behavioral model to provide execution and verification of the global system behavior. In the background, we studied approaches that aim at get the global representation of a heterogeneous system, in both structural and behavioral way. We focused on approaches that have achieved to automate the coordination between behavioral models by capturing coordination patterns at language level. These approaches proposed tools or frameworks that automate the coordination of heterogeneous behavioral models. We have noted that, in these approaches, the knowledge about the coordination is hidden inside a tool, thus limiting reasoning. Moreover, they express the coordination in a GPL thus limiting the verification and validation of the whole system. Based on the state-of-art approaches, we have determined that:
	\begin{itemize}

	\item To capture the specification of coordination patterns, the coordination has to be specified at language level;
	 
	\item The specification of coordination patterns must be supported by a dedicated language for integrators; 
	 
	\item The coordination between models must be generated by using a formal language that allows system designers to provide verification and validation activities.
	
\end{itemize}
	  
To understand how current approaches have achieved to capture a given coordination pattern, we have proposed a framework for the specification of coordination patterns. We have determined three building-blocks: 
	\begin{itemize}
	\item a Language Behavioral Interface, which exposes partial information about the syntax and the behavioral semantics of languages for coordination purpose; 
	
	\item Correspondence Rules, which specify what and when elements from different languages must be coordinated;
	
	\item Coordination Rules, which specify how the selected elements must be coordinated. 
\end{itemize}

In the light of this framework, we compared existing approaches and we stated the requirements to make them more flexible and better customizable. More precisely, we proposed requirements to improve the supporting of heterogenenous languages and the customization of the specification of coordination patterns between them. Based on these requirements, we proposed \bcool, a dedicated language to capture coordination patterns between languages. In our approach, we have proposed a language behavioral interface made of event types, \ie \dse. These event types act as coordination points on the language behavioral semantics. Then, we have proposed to specify coordination patterns by using operators that define a correspondence matching that selects instances of \dse, and a coordination rule that defines how the selected instances of \dse must be coordinated. Using \bcool, the know-how of a system designer is made explicit, stored and shared in libraries. Furthermore, by relying on such specification at language level, a system designer can generate a model of coordination in \ccsl when it is applied between particular models. The use of a formal language to express the coordination allowed us to provide execution and verification of the coordinated system.
	
We have presented the implementation of \bcool as a set of Eclipse plugins into the GEMOC studio. The integrated studio has managed the task of integrators and system designers by providing dedicated workbenches. In the language workbench, an integrator can develop operators between languages. Then, in the modeling workbench, a system designer can use these operators to automate the coordination of models. We have proposed a dedicated language named \bflow that allows a system designer to specify what operators are applied on a set of models. Then, from this specification, a system designer can generate a model of coordination in \ccsl that can be executed and verified.  
	
To validate our approach, we have presented as use case the coordination of the heterogeneous models of a surveillance camera system. We modeled the different parts of the system by using the TFSM and Activity languages. To coordinate these models, we specified in \bcool three coordination patterns that we captured by using three operators. Unlike coordination frameworks in which the semantics of the coordination is hidden, we have explicitly specified these patterns by using a dedicated language. We used this specification to generate the coordination for the surveillance camera system, and then we executed the overall system.


	  
\section{Future Works}
\bcool provides some perspectives to extend and to improve the work carried out in this thesis. We list the propositions we consider essential to the continuity of this work:

\begin{itemize}
	\item \textbf{Extending \bcool to support the coordination of data:} System designers build coordination models to specify how models interact. The interactions between models can rely on events but also on data, \ie data-driven coordination. In this case, a model exchanges data with another model. With \bcool, we have managed interactions that rely on events, \ie control-driven coordination. To support the specification of coordination patterns that involve the exchange of data between models, we have to extend \bcool to support data driven coordination. More precisely, we have to add a way to specify when a value of a variable in a model is carried to a new value in a variable in another model. This makes arise several questions:
			\begin{itemize}
				\item \emph{What} values must be carried, \eg the first one, the last one;
				\item \emph{When} the values of two variables must be synchronized, \eg immediately after one of them changes, after a period;
				\item \emph{How} the values of a variable must be carried from one model to another, \eg the same value, after some conversion. 
			\end{itemize}
	In \bcool, such information should be encoded in a \emph{data coordination rule}. In addition, the current coordination rule should be used to synchronize the events associated with the change of the value of a variable. Thus, the resulting model of coordination would have some \ccsl specification but also some code that represents the exchange of data between models. We have to investigate what language should be used to express such exchanges. %For the first experimentations, Java could be fine.    	 
	\item \textbf{Generalizing the specification of correspondences by using a dedicated language:} In \bcool, the correspondence matching can capture implicit and explicit correspondences between elements of models. Currently, the explicit correspondence is only supported if one of the metamodel is modified to refer concepts from a different metamodel. To avoid such modification, we need a language to specify correspondences between concepts from different languages without modifying the metamodels. Such a work is similar to the concept of user-defined connector types in which a system designer can define a new type of connector and the roles on which it applies. By following this idea, a language dedicated for correspondences should enable the definition of connector types between concepts of the language syntax. Then, in a \bcool specification, the correspondence matching could rely on such an explicit correspondence to select the elements on which apply the coordination rule. This is interesting in the case of allocation in which there is a model of the hardware, a model of the application and a mapping model that is often generated by using some heuristic. Thus, the mapping model can be the input for a \bcool specification to generate the coordination model that represents the deployment between the hardware model and the platform model.
	\item \textbf{Extending \bcool to generate a model of coordination in other language:} \bcool relies on \ccsl to express the coordination. However, we could rely on other coordination language to take advantages of the tooling from other environment. To do so, we have to investigate if the current event relations can be translated to another coordination language.
	
	\item \textbf{Using \bcool for the synthesis of the orchestrator for Co-Simulation of Functional Mock-up Units:} The analysis of Cyber-Physical Systems (CPS) involves the use of components that are described by differential equations but also by using Discrete-Event Simulation (DES) techniques. As a result, several simulation tools and techniques are needed for CPS simulation and analysis. In this context, the Functional Mock-up Interface (FMI)~\cite{fmibib2} is a tool independent standard to support the co-simulation of dynamic models. The FMI standard provides a well defined specification and API to integrate simulation components. One key requirement for Co-Simulation via FMI is to develop a Master Algorithm (MA) that orchestrates the steps of Co-Simulation~\cite{fmibib}. For instance, the MA has to control the data exchange and the time advancement among individual simulations. The FMI standard, however, does not describe or limit the implementation of the MAs. Currently, the MA is specified each time when a particular system has to be co-simulated, which remains tedious and error prone. An interesting future work is to generate the MA from a \bcool specification. Currently, the control and timing coordination is well managed by \bcool. However, to exploit the data exchange proposed by the FMI standard, it is mandatory to extend \bcool to support data-driven coordination. 
	
	\item \textbf{Generalizing the specification of coordination patterns between existing modeling languages:} The development of heterogeneous system is currently done by using different existing modeling languages like Matlab, SDL or Modelica, which are developed by using different technologies. To specify coordination patterns between these languages, we have to investigate how to add support for existing modeling languages in \bcool. Currently, to capture the specification of coordination patterns, we rely on partial information about the syntax and the behavioral semantics of languages that is contained in the language behavioral interface. It is thus mandatory that current modeling languages expose part of its syntax and behavioral semantics. The language behavioral interface could be a standard way to do it. In other words, modeling languages could provide a language behavioral interface for coordination purpose. %We have to investigate if the current language behavioral interface proposed in~\cite{sle13-combemale} is enough to support these languages.    
				
		%\item It is thus mandatory a standard language behavioral interface. Such a language behavioral interface may be generated from code.   
	

	
	

	

	
		
				%\item To analyze CPS, some components can be easily described by differential equations, while others like communication networks require Discrete-Event Simulation (DEVS) techniques. As a result, several simulation tools and techniques are needed for CPS simulation and analysis. In particular, Co-Simulation (Cooperative Simulation) enables system designers to simulate individual components using different simulation tools.   
				
				%\item Cyber-Physical Systems (CPS) are composed of several collaborating physical and computing components that interact through embedded communication capabilities. 
		
		
		
		%\item Theoretically, an MA for direct co-ordination among FMUs can be developed for each distributed simulation problem, but that would be expensive and subject to errors.
		
\end{itemize}
