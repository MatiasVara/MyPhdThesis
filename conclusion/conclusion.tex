\chapter{Conclusion}
\label{ch:conclusions}

\section{Overview}

In this thesis, we have addressed the problem of the coordination of heterogeneous behavioral models by proposing \bcool, a dedicated language for integrators to capture the specification of coordination patterns between languages. 
	
To build \bcool, we have proposed a framework for coordination pattern approaches. We have proposed a language behavioral interface made of event types, \ie \dse. These event types act as coordination points on the language behavioral semantics. Then, we have proposed to specify coordination patterns by using operators that define a correspondence matching that selects instances of \dse, and a coordination rule that defines how the selected instances of \dse must be coordinated. Using \bcool, the know-how of an system designer is made explicit, stored and shared in libraries.

By relying on such specification at language level, a system designer can generate a model of coordination when it is applied between particular models. Furthermore, by relying on \ccsl to express the coordination, we can execute and verify of the coordinated system.   
	
We have shown the current implementation of \bcool into the GEMOC studio. The environment enables an integrator to develop operators between languages. These operators can then be used by a system designer to automate the coordination of models. 
	
We have validated our approach by presenting two examples: the coffee machine and the surveillance camera system. We have used the studio to develop operators and then coordinate the models. The workbench has provided us simulation and verification of the examples. 
	  
\section{Future Works}
\bcool provides some perspectives to extend and to improve the work carried out in this thesis. We list the propositions we consider essential to the continuity of this work:

\begin{itemize}
	\item \textbf{Extending \bcool to support the coordination of data:} System designers build coordination models to specify how models interact. The interactions between models can rely on events but also on data, \ie data-driven coordination. In this case, data from a model is carried to data to another model. With \bcool, we have managed interactions that rely on events, \ie control-driven coordination. To support the specification of coordination patterns that involve the exchange of data between models, we have to extend \bcool to support data driven coordination. More precisely, we have to add a way to specify when a value of a variable in a model is carried to a new value in variable in another model. However, several questions arise like:
			\begin{itemize}
				\item \emph{What} value must be carry, \eg the last one, by using a fifo.
				\item \emph{When} the value of two variables must be synchronized, \eg immediately after one of them changes, every n seconds;
				\item \emph{How} the value of a variable must be carry from one model to another, \eg the same value, with some conversion. 
			\end{itemize}
	In \bcool, such information should be encoded in a \emph{data coordination rule}. In addition, a coordination rule should be used to synchronize the change of one variable to other. It could be done by constraining the corresponding events. The supporting of data driven coordination will enable the specification of much more richen coordination patterns. 	 
	\item \textbf{Generalizing the specification of correspondences by using a dedicated language:} In \bcool, the correspondence matching can capture implicit and explicit correspondences between elements of models. So far, the explicit correspondences need to modify the metamodel of the languages to refer concepts from different languages. To avoid such modification, we need a language to specify correspondences between concepts from languages without modifying the metamodels. Such a work is similar to the concept of user-defined connector types in which a system designer can define a new type of connector and the roles on which it applies. By following this idea, a metalanguage for correspondence should enable the definition of connector types between concepts of the language syntax. Then, in a \bcool specification, the correspondence matching could rely on such a explicit correspondences to select the elements on which apply the coordination rule. This is interesting in the case of allocation. In such a case, there is a model of the hardware, a model of the application and a mapping models. A mapping model is often generated by using some heuristic. These models can be the input for a \bcool operator to generate the coordination between the hardware and the platform model.  
	\item \textbf{Extending \bcool to generate a model of coordination in other language:} \bcool relies on \ccsl to express the coordination. However, we could rely on other coordination language to take advantages of the tooling from other environments. To do so, we have to investigate if the current event relations can be translated to another coordination language.
\end{itemize}
