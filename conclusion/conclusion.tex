\chapter{Conclusion}
\label{ch:conclusions}

\section{Overview}

\begin{itemize}
		
	\item In this thesis, we have addressed the problem of the coordination of heterogeneous behavioral models by proposing \bcool, a dedicated language to capture the specification of coordination patterns between languages. 
	
	
	\item To build \bcool, we have proposed a framework for coordination pattern specification. \bcool is a particular implementation of such a framework.
	
	\item In \bcool, we have proposed to rely on a language behavioral interface made of event types, \ie \dse. These event types act as coordination points on the language behavioral semantics. 
	
	\item We have proposed to specify coordination patterns by using operators. Operators define a correspondence matching that selects instances of \dse, and a coordination rule that defines how the selected instances of \dse must be coordinated.  

	\item We have shown that the specification at language level allows us to generate a model of coordination in \ccsl when it is applied between particular models. 
	
	\item Furthermore, by relying on a formal language to express the coordination, we can provide execution and verification of the coordinated system.   
	
	\item We haven shown the current implementation of \bcool into the GEMOC studio. The environment enables an integrator to develop operators between languages that a system designer can use to coordinate their models. We have validated our approach by two examples: the coffee machine and the surveillance camera system. We have shown the use of the studio to develop operators, to coordinate models and to execute and validate the coordinated system. 
	  

	
\end{itemize}
\section{Future Works}

We list the propositions we consider essential to the continuity of this work:

\begin{itemize}
	\item \textbf{Extending \bcool to support the coordination of data:} 
	\begin{itemize}
		\item System designers build coordination models to specify how models interact. This interaction can rely on events (\ie control-driven coordination) but also on data (\ie data-driven coordination). Data-driven coordination specifies interaction between models that relies on data, \eg send and receive mechanisms. To support the specification of coordination pattern that involve data between models, we have to extend \bcool to support data driven coordination.
	
	\end{itemize}
	
	\item \textbf{Generalizing the specification of correspondences by using a dedicated language:}
		\begin{itemize}
			\item In \bcool, the correspondence matching can capture implicit and explicit correspondences between elements of models. In particular, explicit correspondences need to modify the metamodel of the languages. To avoid this, there is a need of a language to specify correspondence between elements without modifying the metamodels of the languages. Then, a \bcool specification could rely on such a metamodel of correspondence to select the elements on which apply the coordination rule. 
			\item To implement this, it is necessary to modify the current metamodel of \bcool by adding a reference in the correspondence matching to an EObject. Then, such a reference could be use to refer to connectors. 
		\end{itemize}
			\item \textbf{Extending \bcool to generate a model of coordination in other language:} 
			\begin{itemize}
				\item \bcool relies on \ccsl to express the coordination. However, we could rely on other coordination language like Rapide to take advantages of the tooling from other environments.
				\item However, it is necessary to investigate which coordination could be used.         
			\end{itemize}
\end{itemize}
