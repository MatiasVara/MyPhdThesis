\chapter{Conclusion}
\label{ch:conclusions}

\section{Overview}

\begin{itemize}
	
	\item Currently, the developing of complex application involve the developement of heterogeneous behavioral models. 
	
	\item To get the emerging system behavior, the behavior of this models has to be coordinated. 
	
	\item In this thesis, we have addressed this problem by proposing to specify coordination patterns between languages. 
	
	\item Coordination patterns capture the systematic activity of a system designer that relies on his know-how to coordinate the behavior of models.


	\item We proposed a framework to capture the specification of coordination patterns. We have identified three builder elements to specify a coordination pattern:
		\begin{itemize}
			\item a language behavioral interface
			\item a correspondence rule
			\item a coordination rule
		\end{itemize}
	\item By relying on this framework, we compared current coordination frameworks and we proposed the structure of a language to specify coordination pattern, \bcool. 
	  	
	\item \bcool is dedicated language to specify coordination patterns between languages.
	  	
	
	\item We have validated our approach by applying \bcool to define set of operators between tfsm and activity.  

	
\end{itemize}
\section{Future Work}
\begin{itemize}
	\item metalanguage of correspondences
	\item data
	\item properties 
	\item coordination patterns like map reduce? 
\end{itemize}
