\section{Introduction}
To deal with complexity issues in the development of applications for complex systems, Model Driven Engineering (MDE) proposes to rely on \emph{Models}. A model is an abstraction of the real world made in order to facilitate an understanding of the way it works. In the context of MDE, a software model enables a developer to reduce the complexity of an application by ignoring non-essential details. This enables the developer to reason about the application. 

The Object Management Group (OMG) proposes to specify models by relying on a language that has a well-defined form (syntax), meaning (semantics) and possible rules of analysis, inferences or proof for its constructs~\cite{mdaguide}. Thus, to build models, MDE proposes Domain Specific Modeling Languages (DSMLs). They are built by \emph{Language Engineers} to describe the structure but also the behavior of a particular domain. As a result, a DSML is defined by a syntax and a semantics. The syntax is described by a metamodel that defines the concepts and relations that the language is made up. A metamodel is a model that is developed by using a metameta language, \eg MOF, ECORE. To define the semantics, the language theory proposed three types of semantic definitions: Operational~\cite{operationbib}, Axiomatic~\cite{axiomaticbib} and Translational~\cite{translationalbib}. The concurrent theory has also proposed other ways to describe the behavior of a model. This behavior is characterized by the so-called Models of Computation (\mocc)~\cite{moccsemanticbib}. In this thesis, we focus on this approach for the description of the behavioral semantics of a language.


Based on a DSML, a domain expert builds a model to describe the structure and the behavior of a domain. However, the development of complex applications is often tackled by several domain experts. Each domain expert uses its own DSML to describe a part of the system. Thus, the use of several DSMLs results in a heterogeneous specification, \ie made of models that conform to different DSMLs. 

At some point of the development, a global representation of the system is needed to reason about the system as a whole. For instance, a system designer must be able to perform verification and validation activities of the overall system. Thus, it is necessary to specify how models and languages are related in a structural and behavioral way. 

This chapter presents the state-of-art approaches that give support for the heterogeneous development of systems by providing composition and/or coordination of models/languages. We begin by presenting \emph{Composition Approaches} that propose to compose models/languages to obtain a new model/language. We continue by presenting \emph{Coordination Approaches} that propose to specify the interaction between model/languages into an additional model so-called \emph{Model of Coordination}.   


%The development of complex software intensive systems involves interactions between different subsystems. For instance, embedded and cyber-physical systems require the interaction of multiple computing resources (general-purpose processors, DSP, GPU), and various digital or analog devices (sensors, actuators) connected through a wide range of heterogeneous communication resources (buses, networks, meshes).
%To deal with complexity issues in the development of applications for complex systems, Model Driven Engineering (MDE) proposes to rely on \emph{Models}. A model is an abstraction of the real world made in order to facilitate an understanding of the way it works. In the context of MDE, a software model enables a developer to reduce the complexity of an application by ignoring non-essential details. In this sense, The Object Management Group (OMG) defines that \emph{a model is a representation of a part of the function, structure and/or behavior of a system. The model specification is based on a language that has a well-defined form (syntax), meaning (semantics) and possible rules of analysis, inferences or proof for its constructs”}. This enables the developer to reason about the application. To built models, MDE proposes \emph{Domain Specific Modeling Languages} (DSMLs). They are built by an expert to describe the structure but also the behavior of a domain. As a result, a DSML, as any other language, is defined by a syntax and a behavioral semantics.  

%The development of complex applications may involve several DSMLs. Each expert uses its own language to describe its domain. This results in a heterogeneous specification, \ie made of models that conform to different DSMLs. Furthermore, nowadays systems are more and more coupled. As a result, system designers need to get a global representation to reason about the system as a whole. For instance, a system designer must be able to perform verification and validation activities of the overall system. To get a global representation, system designers need to combine models in a structural and behavioral way. In this thesis, we use the word \emph{composition} to refer to the combination of structural models/languages to obtain a new structural model/language. Conversely, we use the word \emph{coordination} to refer to the specification of the interactions between the behaviors of models/language into a new model, a so-called \emph{Model of Coordination}.   
 
%In this chapter, we first present an overview of composition approaches, which propose to compose the structure of models/languages to obtain a new model/language. Afterwards, we present state-of-the-art approaches that coordinate the behavior of languages/models. In these latter approaches, the interactions between languages/models are specified as a new model. By relying on these approaches, we conclude this chapter with the requirements for our language \bcool.        
