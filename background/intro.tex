\section{Introduction}
The development of complex software intensive systems involves interactions between different subsystems. For instance, embedded and cyber-physical systems require the interaction of multiple computing resources (general-purpose processors, DSP, GPU), and various digital or analog devices (sensors, actuators) connected through a wide range of heterogeneous communication resources (buses, networks, meshes).

To tackle the development of application for complex systems, Model Driven Engineering (MDE) proposes to rely on \emph{Models}. A model is an abstraction of the real word made in order to facilitate an understanding of its workings. In the context of MDE, a software model enables a developer to reduce the complexity of an application by ignoring non-essential details. In this sense, The Object Management Group (OMG) defines that \emph{a model is a representation of a part of the function, structure and/or behavior of a system. The model specification is based on a language that has a well-defined form (syntax), meaning (semantics) and possible rules of analysis, inferences or proof for its constructs”}. This enables the developer to reason about the application. To built models, MDE proposes \emph{Domain Specific Modeling Languages} (DSMLs). They are built by an expert in order to describe the structure but also the behavior of a domain. As a result, a DSML, as any other language, is defined by a syntax and a behavioral semantics. 

The developing of complex application may involve the use of several DSMLs. Each expert uses its own language to describe its domain. This results in a heterogeneous specification, \ie made of models that conform to different DSMLs. Furthermore, nowadays systems are more and more coupled. As a result, system designers need to get a global representation to reason about the system as a whole. For instance, a system designer must be able to perform verification and validation activities of the overall system. To get a global representation, system designer needs to combine models in a structural and behavioral way. In this thesis, we use the word \emph{composition} to refer the combination of structural models/languages in order obtain a new structural model/language. Conversely, we use the word \emph{coordination} to refer the specification of the interaction between the behavior of models/language into a new model so-called \emph{Model of Coordination}.   
 
In this chapter, we first present an overview of composition approaches. They propose to compose the structure of models/languages in order to obtain a new model/language. Afterwards, we present the state of art approaches that coordinate the behavior of languages/models. In these approaches, the interaction between languages/models is specified into a new model. By relying on these approaches, we conclude this chapter with the requirements for our language \bcool.        
