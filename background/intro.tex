\section{Introduction}
To develop application for complex systems, no unique language is able to deal satisfactory with all the facets of developing. Model Driven Engineering proposes to use a specific language for each domain. Thus, the design of complex systems often relies on several Domain Specific Modeling Languages (DSMLs) that may pertain to different theoretical domains with its own expressiveness and properties. As a result, several models conforming to different DSMLs are developed and the specification of the overall system becomes \emph{heterogeneous}. 
	
A DSML, as any other language, is defined by a \emph{syntax} and a \emph{semantics}. The syntax specifies the concepts of the language and their relationships. The semantics describes the expected evolutions(s) of the model state during its execution. To understand the system and its emerging behavior globally, it is necessary to specify how models and languages are related to each other, in both a structural and a behavioral way. 
	
In this chapter, we first overview the current support for the syntactic composition of models and languages. We then focus on the state of art approaches that deal with the composition and coordination of behavioral models in order to obtain the global emerging behavior.  