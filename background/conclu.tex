\section{Conclusion}
In this chapter, we have presented a background of current approaches that deal with the integration of heterogeneous models and languages. 

We have presented approaches that compose heterogeneous models/languages in order to obtain a new model/language. The composition of models has been automated by looking for correspondences between heterogeneous model, and then composing them into a new model that can conform to other language. 
While most of the approaches only consider structural correspondences, only a few consider also the behavior of model to find similarities. Furthermore, we have determined that these approaches only compose homogeneous models. This limits the use of such approaches in complex systems where heterogeneous behavioral models may be used.

Unlike the composition of models, we have noted that the composition of languages has not been automatized. Thus, a system designer has to 1) find correspondences between concepts of different languages and 2) specify how these concepts are related. This results in a new language. The inputs languages and the composed language are conformed to the same meta metamodel. In most of these approaches, the correspondence are only between syntactic elements, and only one approach enables to define behavioral correspondences in order to obtain a new behavioral semantics.  

We want to highlight that an specification at language level (as proposed by composition models approaches) is easier to handle than a unified language (as proposed by composition of languages Approaches). From our point of view, Composition of Languages approaches are not suitable for separation of preoccupation and development of a single system by various domain experts that focus on a specific part of the system.

Afterwards, we have presented the state of art approaches that deal with coordination of the behavior of heterogeneous models/languages. First, we have presented the work done by Coordination Languages and ADLs that support the coordination of heterogeneous models. However, the system designer has to manually specify each relation, thus making this task tedious and error prone. Then, we have shown how coordination pattern approaches have leveraged on the know-how of system designer to automate the coordination between models. However, we have noted that the knowledge about system integration is encoded within a framework. Furthermore, in frameworks, the model of coordination is expressed by using a general purpose language thus limiting verification and validation activities. 

%We want to highligne that the current integration of languages is based on two mechanism: composition and coordination. While composition tends to provide to compose language into a new language, Coordination tends to provide a model of coordination which is external to the models itself. It is interesting that coordination only considers behavioral while composition only structural. Why? 
In this thesis, we propose to capture explicitly the know-how of a system designer by using \bcool, a dedicated language to capture coordination patterns between languages, thus reifying coordination at the language level (see Figure~\ref{fig:bcoolapp}). From this specification, a formal model of coordination is generated. This enables validation and verification of the coordinated system. 

In the next chapter, we focus on coordination pattern approaches to understand how a given coordination pattern is being captured. We leverage on such knowledge to built \bcool. 

\begin{figure}
\begin{center}
\includegraphics[width=0.7\textwidth]{background/figs/bcoolapp}
\caption{Overview of Our Approach}
\label{fig:bcoolapp}
\end{center}
\end{figure}	


%\begin{landscape}% Landscape page
%\begin{table}[]
	%\centering
	%\caption{Overview of Behavioral Composition/Coordination Approaches}
	%\label{tbl:overview}
	%\resizebox{\textwidth}{!}{%
		%\begin{tabular}{@{}|c|l|l|c|c|c|c|c|c|c|
			%	>{\columncolor[HTML]{9AFF99}}c |@{}}
			%\toprule
			%\multicolumn{3}{|c|}{} & \multicolumn{2}{c|}{Behavioral Composition Approaches} & \multicolumn{3}{c|}{Coordination %of Model Approaches} & \multicolumn{2}{c|}{Coordination Pattern Approaches} & Our Approach \\ \cmidrule(l){4-11} 
			%\multicolumn{3}{|c|}{\multirow{-2}{*}{}} & Semantic Anchoring & \cite{compostatechartsbib} & Esper & Rapide & BIP & %Ptolemy & Di Natale et. al & BCOoL \\ \midrule
			%\multicolumn{3}{|c|}{Number of Languages/Models Supported} & 1 & 1 & N & N & 1 & Predefined set & 2 & N Languages %\\ \midrule
			%\multicolumn{3}{|c|}{Specification of the Coordination (from a system designer point of view)} &  & Implicit & %Explicit & Explicit & Explicit & Implicit & Impicit & Explicit \\ \midrule
			%\multicolumn{3}{|c|}{Glue (coordination at model level)} &  &  & Formal & Formal & Formal & Java (generated) & C++ %(generated) & Formal (generated) \\ \bottomrule
%		\end{tabular}
%	}
%\end{table}
%\end{landscape}
