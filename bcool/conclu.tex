\section{Conclusion}
\begin{itemize}
	
	\item In this chapter, we have presented \bcool, a dedicated (meta)language that enables integrators to capture the specification of coordination patterns between heterogeneous languages. With \bcool, integrators specify coordination patterns by using operators that specifies what, when and how \dse from different language behavioral interfaces must be coordinated. Using \bcool, the know-how of an integrator is made explicit, stored and shared in libraries.

	\item Afterwards, we have show how a system designer can use such specification to automate the coordination between models. Additionally, since we rely on a formal language to express the coordination, a system designer can verify and validate the coordinated system.  
	
	\item We have illustrated our approach by defining a simple operator between the TFSM and Activity languages. Then, we have generated the model of coordination for the heterogeneous model of a coffee machine. 
	
	\item We have shown the current integration of \bcool into the GEMOC studio that provides a language workbench and a modeling workbench. We have presented the language workbench in which an integrator can develop \bcool operators. Then, in the modeling workbench, these operators can be used by a system designer to coordinate and animate their models. Furthermore, the modeling workbench provides tools to analize the coordinated model.  
	
	\item We have compared our approach with coordination frameworks and coordination languages.    

	\item In the next chapter
\end{itemize}
