\section{Conclusion}
In this chapter, we have presented \bcool, a dedicated (meta)language that enables integrators to capture the specification of coordination patterns between heterogeneous languages. \bcool is a particular implementation of the framework presented in Chapter~\ref{ch:framework}. To illustrate the syntax and the semantics of \bcool, We defined a simple operator between the TFSM and Activity languages. We have presented the current implementation of \bcool into the GEMOC studio that provides a language workbench and a modeling workbench. In the language workbench, a language integrator can develop \bcool operators. Then, in the modeling workbench, a system designer can use these operators to coordinate and execute their models. Furthermore, the modeling workbench provides tools to analyze the coordinated model. We have shown the use of the language workbench by developing the running example operator, and then, we have shown the use of the modeling workbench by coordinating and verifying the heterogeneous model of the coffee machine. To finish the chapter, we have compared \bcool with coordination frameworks and coordination languages by relying on four criteria. From this analysis, we have determined that our approach makes a clear separation between the task of a language integrator and a system designer. Furthermore, this separation is supported by dedicated workbenches. In the next chapter, we propose to validate our approach by using as use case the coordination of the heterogeneous models of a surveillance camera system. We rely on the integrated studio to develop a set of coordination operators, and then, we use them to coordinate and execute the system. 


%We have noted that our approach is the only that makes a clear separation between the role of an integrator and the role on an system designer by providing dedicated environment for each one. 
%With \bcool, integrators specify coordination patterns by using operators that specifies what, when and how \dse from different language behavioral interfaces must be coordinated. 