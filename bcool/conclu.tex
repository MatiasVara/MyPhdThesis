\section{Conclusion}
\begin{itemize}
	
	\item In this chapter, we have presented \bcool, a dedicated (meta)language that enables integrators to capture the specification of coordination patterns between heterogeneous languages. Using \bcool, the know-how of an integrator is made explicit, stored and shared in libraries.

	\item Afterwards, we have show how a system designer can use such specification to automate the coordination between models. Additionally, since we rely on a formal language to express the coordination, a system designer can verify and validate the coordinated system.  
	
	\item We have illustrated our approach by defining a simple operator between the TFSM and Activity languages. Then, we have generated the model of coordination for the heterogeneous model of a coffee machine. 
	
	\item We have shown the \bcool framework which is the integration of \bcool into the GEMOC studio that provides a language workbench and a modeling workbench. We have presented the language workbench in which an integrator can develop \bcool operators. Then, in the modeling workbench, a system designer can use these operators to coordinate and execute their models. Furthermore, the modeling workbench provides tools to analyze the coordinated model.  
	
	\item Afterwards, we have presentd some critierias to compare our approach with coordination frameworks and coordination languages. We have noted that our approach is the only that makes a clear separation between an integrator and a system designer. While coordination framework the role of the integrator is limited, in corodination langages he does exist.
	\item coordination languages and coordination frameworks are tools for system designer as weel the modeling workbench presented in our approach. 
	\item In our approach, we have dealt with both stakeholders by presenting dedicated environment for each one.  
	
	\item We have compared our approach with coordination frameworks and coordination languages.    

	\item In the next chapter
\end{itemize}


%With \bcool, integrators specify coordination patterns by using operators that specifies what, when and how \dse from different language behavioral interfaces must be coordinated. 