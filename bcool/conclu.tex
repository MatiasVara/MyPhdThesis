\section{Conclusion}
In this chapter, we have presented \bcool, a dedicated (meta)language that enables integrators to capture the specification of coordination patterns between heterogeneous languages. Using \bcool, the know-how of an integrator is made explicit, stored and shared in libraries. Afterwards, we have show how a system designer can use such specification to automate the coordination between models. Since our approach relies on a formal language to express the coordination, a system designer can verify and validate the coordinated system. We have illustrated our approach by defining a simple operator between the TFSM and Activity languages. Then, we have generated the model of coordination for the heterogeneous model of a coffee machine. We have presented the \bcool framework which is the integration of \bcool into the GEMOC studio that provides a language workbench and a modeling workbench. We have presented the language workbench in which an integrator can develop \bcool operators. Then, in the modeling workbench, a system designer can use these operators to coordinate and execute their models. Furthermore, the modeling workbench provides tools to analyze the coordinated model. Afterwards, we have compared \bcool with coordination frameworks and coordination languages. We have noted that our approach is the only that makes a clear separation between the role of an integrator and the role on an system designer by providing dedicated environment for each one. To validate our approach, we present in the following chapter the development of some \bcool coordination operators. We then use these operators to generate the coordination model for a video surveillance system. The modeling workbench is finally used to execute and validate the result.

%With \bcool, integrators specify coordination patterns by using operators that specifies what, when and how \dse from different language behavioral interfaces must be coordinated. 