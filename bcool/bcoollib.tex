\subsection{Library}
\label{subsec:bcoollib}

Libraries gather some predefined event expressions and relations, which must be imported by the specification (\emph{ImportedLibStatement} in Figure~\ref{fig:bcool}). They are organized by modeling domains to gather all the relevant event relations and expressions. \bcool relies on \moccml~\cite{moccmlbib} (Model of Concurrency and Communication Language) that allows the definition of libraries of constraints between events.

Roughly speaking, a \moccml library is a set of declarations together with their formal parameters. A library also contains some definitions, which give the actual behavior of the declarations. Event expressions create a new event from their parameters (\eg building the \textit{Union}, or the \textit{Intersection} of its parameters). Such constraints are used in \bcool either to provide global events used in different operators or to define some local events used in the coordination rules. Relations, however, constrain the evolution of the events given as formal parameters. In \moccml, relations can be defined by following the syntax of \ccsl~\cite{ccslbib}. In addition, the language gives state based support for the definition of relations thus enabling to define complex communication protocols, \eg the AMBA protocol~\cite{ambabus}.

\todo{To show a graphical representation of relation in \moccml by using a TFSM}


\begin{lstlisting}[language=moccml,
		caption={Facilities Library in \moccml},
		label={lst:moccmllib}, 
		basicstyle=\scriptsize\ttfamily, backgroundcolor=\color{LGrey}, numbers=left, xleftmargin=2pt]
		AutomataConstraintLibrary Facilities{ 
		import "platform:/plugin/fr.inria.aoste.timesquare.ccslkernel.model/ccsllibrary/kernel.ccslLib" as kernel;
		import "platform:/plugin/fr.inria.aoste.timesquare.ccslkernel.model/ccsllibrary/CCSL.ccslLib" as CCSLLib;
		
		RelationLibrary CCSLBasedBCOoLRelations {
		RelationDefinition RendezVousDef[RendezVous]{ 
			Relation RendezVousrelat[Coincides]( Clock1->ClockA, Clock2->ClockB)
		}
		RelationDefinition RendezVouswithGlobalClockDef[RendezVouswithGlobalClock]{ 
			Expression sampledStartAction = SampledBy (Sampled -> globalClock, Trigger-> startAction)
			Expression sampledOccurs = SampledBy (Sampled -> globalClock, Trigger-> occurs)
			Relation RendezVousSamples[Coincides](   Clock2-> sampledStartAction , Clock1-> sampledOccurs  )
		}
		RelationDeclaration RendezVous (ClockA:clock, ClockB:clock) 
		RelationDeclaration RendezVouswithGlobalClock (startAction: clock , occurs: clock, globalClock: clock )
		}
		ExpressionLibrary BCOolExpressions{
			ExpressionDefinition SampledDef [SampledBy]{
				root=Sample_result 
				Expression Sample_result = SampledOn(SampledOnTrigger -> Trigger, SampledOnSampledClock ->Sampled)
		}
		ExpressionDeclaration SampledBy(Trigger: clock,  Sampled: clock):clock
		}	
		}
\end{lstlisting}

\bcool includes the library \emph{facilities.moccmllib} that provides all the declarations used in the previous examples. This is partially shows in Listing~\ref{lst:moccmllib}. The integrator however can extend the current library by defining new specific constraints depending on its problems and domain. For instance, the coordination rule presented in Listing~\ref{lst:bcoolrunningexample2} can be rewritten by defining an event relation named \emph{RendezvousWithGlobalClock} (Listing~\ref{lst:bcoolrunningexamplellib}: line 11). Then, such a relation can be reused in other specification. Furthermore, libraries improves the readability of a \bcool specification by gathering domain specific relations. 
	
	\begin{lstlisting}[language=bcool,
	caption={Synchronized product operator between the TFSM and Activity languages by using the library},
	label={lst:bcoolrunningexamplellib}, 
	basicstyle=\scriptsize\ttfamily, backgroundcolor=\color{LGrey}, numbers=left, firstnumber=10, xleftmargin=2pt]
	CoordinationRule: 
		RendezVouswithGlobalClock(dse1, dse2, globalClock)
	end operator
	\end{lstlisting}
	
%By relying on \moccml, the application of \bcool operators generate a coordination specification in \ccsl. We can then use \ccsl tool (TimeSquare~\cite{timesquarebib}) to perform analyze and execute the generated coordination specification. This is further discussed in Section~\ref{sec:validation}.