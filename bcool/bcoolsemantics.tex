\section{Execution Semantics}
In this section, we describe the execution semantics of \bcool, \ie how a \bcool specification is used to generate a coordination model. To illustrate the different steps in the generation, we rely on the application of the running example operator (see Listing~\ref{lst:bcoolrunningexample}) between the models of the coffee machine.

\begin{figure} [h]
	\center
	\includegraphics[width=.9\textwidth]{bcool/figs/semantics}
	\caption{Steps in the application of the \bcool specification between the models of the coffee machine}
	\label{fig:semantics}
\end{figure}

Let $Ev$ be the (finite) set of event type names (representing the \dse). Considering a language $L$, A behavioral interface $i_L$ is a subset of event type names, $i_L \subset Ev$. A \bcool specification imports $N$ disjoint language interfaces and a set of operators $\mathcal{O}p$, with $N\geq 2$. Each operator from $\mathcal{O}p$ has a set of formal parameters $\mathcal{P}$, where each parameter is defined by a name and its type (\ie an event type). Each operator also has a correspondence matching condition (denoted $CMC$) and a correspondence rule (denoted $CR$). A \bcool specification is applied to a set of input models denoted $\mathcal{M_I}$, with $|\mathcal{M_I}| = N$.

From an operational point of view, the first step consists in producing the model behavioral interface of each input model. It results in a set of model interfaces denoted $\mathcal{I_{M_I}}$, of size $N$. An interface is a set of events, each of which is typed by an event type. For instance, for the models of the coffee machine, the model behavioral interface of the TFSM \emph{CoffeeCoin} and the Activity \emph{CoffeeAlgorithm} are extracted. This results in two sets of \mse named  $\mathcal{I_M}{1}$ and $\mathcal{I_M}{2}$ (step 1 Figure~\ref{fig:semantics}).

Each operator $op$ in $\mathcal{O}p$ is processed individually and several times with different actual parameters, which depend on the model interfaces in $\mathcal{I_{M_I}}$. The set of actual parameters to be used is obtained by a \emph{restricted} Cartesian product of all the model interfaces in $\mathcal{I_{M_I}}$. The restriction consists in two steps: First, a new set of model interface (denoted $\mathcal{I^{'}_{M_I}}$) is created. For each parameter $p$ in $\mathcal{P}$, a new model interface $\mathcal{I^{\textit{\text{p}}}_{M_I}}$ is created and all the events in $\mathcal{I_{M_I}}$ that have the same type than $p$ are collected in $\mathcal{I^{\textit{\text{p}}}_{M_I}}$. Then, $\mathcal{I^{\textit{\text{p}}}_{M_I}}$ is added to $\mathcal{I^{'}_{M_I}}$ (step 2 in Figure~\ref{fig:semantics}). For instance, the running example operator has as parameters the event types \emph{Action::executeIt} and \emph{FSMEvent::occurs}. Thus, the set $\mathcal{I^{'}_{M_I}}$ is composed of two set named $\mathcal{I_M}{1}{executeIt}$ and $\mathcal{I_M}{1}{occurs}$ (step 2 Figure~\ref{fig:semantics}).
%

Second, a classical Cartesian product is applied on $\mathcal{I^{'}_{M_I}}$. It results in a set containing the list of actual parameters to be used with the operator, \ie each set in the result of the Cartesian product represents the actual parameters of the operator (step 3 in Figure~\ref{fig:semantics}). For each set $actualParams$ in the result of the Cartesian product, if $actualParams$ satisfies the correspondence matching condition ($CMC$), then the coordination rule ($CR$) is instantiated with the values in $actualParams$. Returning to the running example operator, the correspondence matching condition is used to select \mse by comparing theirs names. This results in two selected sets (step 4 in Figure~\ref{fig:semantics}). The instantiation is made in two steps. First, the local events, if any, are created in the targeted coordination language according to the expression used to initialize it. The expression can use any event in $actualParams$ and possibly some constants (\eg some Integer constants). The local events are added to $actualParams$ so that they can be used in the next. The second step is the application of the relation. It results in the creation of the corresponding relation in the targeted coordination language. The actual parameters of the coordination rule are then the ones from $actualParams$ or some constants, like for the expressions. For the coffee machine, the event relation rendezvous is instantiated twice; one time for each set in $actualParams$ that satisfies the $CMC$ (step 5 in Figure~\ref{fig:semantics}).

Currently, the application of a \bcool operator generates a \ccsl specification that represents the coordination. Listing~\ref{lst:runningexampleccsl} shows the partial \ccsl specification for the coffee machine. The specification begins by importing the \ccsl specification of each model (Listing~\ref{lst:runningexampleccsl}: line 3 and 4). Then, the main block contains the coordination specification that is made of two instances of the event relation rendezvous (Listing~\ref{lst:runningexampleccsl}: line 9 and 12). Notice that individual specification of each model remains untouched thus not altering the behavior of individual models. The coordination adds some constrains thus restricting the behavior of model, but it does not add new behaviors. This results in a generated coordination that is not intrusive (\ie exogenous).

In the next section, we present the \bcool framework which is an Eclipse environment that enables the developing of \bcool operator between languages and automate the coordination between language. 




\begin{lstlisting}[language=moccml,
caption={Resulting \ccsl specification for the running example},
label={lst:runningexampleccsl}, 
basicstyle=\scriptsize\ttfamily, backgroundcolor=\color{LGrey}, numbers=left, xleftmargin=2pt]
ClockConstraintSystem TFSMandActivity {
imports {
import "coffeeCoin.extendedCCSL" as coffeeCoin;
import "coffeeAlgorithm.extendedCCSL" as coffeeAlgorithm;
}
entryBlock mainBlock
	Block mainBlock {
		Block coffeeCoincoffeeAlgorithmsublock {
			Relation  SyncFSMEventsAndActionselectCoffee_executeItselectCoffee_occurs [ RendezVous ]
			( ClockA -> "coffeeAlgorithm::selectCoffee_executeIt", ClockB -> "coffeeCoin::selectCoffee_occurs")
			Relation  SyncFSMEventsAndActionsreleaseCoffee_executeItreleaseCoffee_occurs [ RendezVous ]
			( ClockA -> "coffeeAlgorithm::releaseCoffee_executeIt", ClockB -> "coffeeCoin::releaseCoffee_occurs")}}
}
\end{lstlisting}    