\section{Use Case: Coordination of the Heterogeneous Model of a Surveillance Camera System}
This section presents the heterogeneous model of a surveillance camera system (Figure~\ref{fig:camerasystem}). To model different parts of the system, we use the TFSM and Activity languages. In the following, we first present the different subsystems, and then, we propose a set of coordination pattern to coordinate the behavior of the different models. 

The video surveillance system is composed of a camera and a battery control. The camera takes pictures by using either the \emph{JPEG2000} or \emph{JPG} algorithm and is powered by a battery. Roughly speaking, the JPEG\footnote{http://www.jpeg.org} algorithm encodes a picture by grouping it in blocks which are transformed by a \textit{forward transformation}. Each block of pixels is transformed to frequency coefficient using either a \textit{Fourier transform} (JPEG) or \textit{Wavelet transform} (JPEG2000). The transformed blocks are \textit{quantized} and then passed to a \textit{Run-Length coder} which compress the data. At the end, the block is \textit{transmitted}. 

In the camera, when the battery is low (\emph{BatteryisLow:executeIt} happens), the battery control makes the camera use the \emph{JPG} algorithm, thus reducing the quality of the picture but also the energy consumption~\cite{encodingcomparison}. When the battery is high (\emph{BatteryisHigh:executeIt} happens), the JPEG2000 algorithm is used instead. In Figure~\ref{fig:camerasystem}, the activity diagrams named \emph{BatteryControl} represents the simple algorithm implemented in the battery control. At the bottom of Figure~\ref{fig:camerasystem}, the TFSM named \emph{CameraControl} represents a partial view of the camera. When the TFSM model is in state \emph{BatteryHigh}, the JPEG2000 algorithm is used (specified by the activity diagram on the right of Figure~\ref{fig:camerasystem} named \emph{doJPEG2000}). When in state \emph{BatteryLow}, the encoding algorithm is replaced by a mere JPEG algorithm represented by an activity named \emph{doJPEG}. The transition from one state to another is done when either the \emph{BatteryIsHigh:occurs} event or the \emph{BatteryIsLow:occurs} event occurs, depending on the current state.	   

\begin{figure}
	\center
	\includegraphics[width=1\columnwidth]{examples/figs/picmodels.pdf}
	\caption{Hierarchical model of a surveillance camera system and a partial representation of the behavioral interface}
	\label{fig:camerasystem}
\end{figure}

To represent the global behavior of the surveillance camera system, we have to specify how these models are coordinated. First, the activity BatteryControl and the TFSM Camera control are coordinated by relying on Actions and FSMEvents. More precisely, the execution of the actions \emph{BatteryisHigh} and \emph{BatteryisLow} must be synchronized with the occurrences of the FSMEvents \emph{BetteryisHigh} and \emph{BatteryisLow}, \eg synchronization between the \mse \emph{BatteryisHigh:execution} and \emph{BatteryisHigh:occurs}. Thus, to coordinate the BatteryControl and the CameraControl, we propose to reuse the operator \emph{SyncFSMEventsAndActions} (see Listing~\ref{lst:bcoolrunningexample}: line 5), which coordinates instances of \dse \emph{occurs} and instances \dse \emph{executeIt} by relying on their names. 

In addition, the execution of the states of the TFSM CameraControl must be coordinated with the activities doJPEG and doJPEG2000. To coordinate these models, we have to specify the coordination between the entering and leaving of a state with the execution of an activity. We chose the semantics in which entering a specific state of a TFSM model triggers the execution of a given Activity. Then, when leaving a state, several semantic variation points may be chosen. The outgoing transitions from a state can be considered, for instance, as preemptive for the activity model (\ie firing a transition from a state to another preempts the internal activity). Alternatively, the transition can be considered as non-preemptive (\ie the states cannot be left before the associated activity finishes). In our case, we chose non-preemptive transitions. This results that, when the states are reached, the corresponding activity is executed. Then, the state can be left only if the activity has finished. 

To coordinate the models TFSM CameraControl and the activities, it is also necessary to specify how the time in the TFSM elapses during the execution of the activities. We propose a coordination that only considers the timing aspects. In these languages, the time is represented differently. In the TFSM language, each state machine has a \emph{localClock} used to measure the time while the Activity language is untimed. The local clock is a \emph{FSMClock}, which defines a \dse named \emph{ticks} whose occurrences represent a physical time increment. In the Activity language, the duration of activities can be represented as the time between the \dse \emph{startActivity} and \dse \emph{finishActivity}. Thus, to coordinate the time, it is necessary to specify the number of \emph{ticks} of the local clock between the occurrence of the \dse \emph{startActivity} and \emph{finishActivity}, \eg between the \mse \emph{doJPEG2000:startActivity}, \emph{doJPEG2000:finishActivity} and \emph{ms:ticks}.  We propose to enforce the execution of the ``internal'' activity to be atomic with respect to the time in the TFSM model. As a result, there is no occurrence of the \dse ticks of the corresponding local clock during the execution of the activity.  

In this section, we have presented the model of a surveillance camera system. To model the system, we used TFSM and Activity languages thus resulting in a heterogeneous system. To coordinate these models, we have described three coordination patterns between the TFSM and Activity languages:
	\begin{itemize}
		\item 1. A coordination pattern that coordinates Actions and FSMEvents, \ie \emph{SyncFSMEventsAndActions} operator. 
		\item 2. A hierarchical coordination pattern that coordinate the entering and leaving of states and the execution of Activities. 
		\item 3. A timing hierarchical coordination pattern that coordinates the time between TFSMs and Activities. 
	\end{itemize}
	
In the following section, we capture the specification of these coordination patterns by using \bcool.  

\todo{I am here}
