\section{Discussion}
In \bcool, the definition of a hierarchical coordination between the TFSM and Activity language is based on operators. More precisely, coordination rules explicitly define the semantics of the resulting hierarchical coordination. Thus, an integrator can vary the semantics of the coordination by only modifying the event relation used in the coordination rule. For instance, in the operator \emph{startActivityWhenEnter}, by slightly modifying the event relation, the transition may become preemptive. Furthermore, we ease the definition of relations by using \moccml, a dedicated language to express constraints between events. Frameworks like Ptolemy do not support such a variation without changing the current implementation. This means modifying the current implementation of a \emph{director} written in Java, which needs a good knowledge of the framework. In our approach, we are using a language dedicated to integrator experts thus easing the understanding and adaptation of the \bcool specification.
		
The \bcool specification allowed us to automate the coordination of the models of a surveillance camera system. In this example, the application of the operator generates six \ccsl relations. By manually coordinating the models (when using a coordination language), this would require specifying each relation manually. The reader can notice that the number of relations increases with the number of model elements involved in the coordination. For instance, for a system with N cameras, the integrator would need to specify 6*N relations, thus making this task tedious and error prone. With \bcool, we have leveraged this task for the integrator at the language level to automate such a task. In addition, by relying on \ccsl, we can provide execution and verification of the coordinated system. In coordination frameworks, these tasks are limited since they express the coordination in a gpl. 