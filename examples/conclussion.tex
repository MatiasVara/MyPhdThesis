\section{Conclusion}
In this chapter, we have validated the use of \bcool by defining three operators between the TFSM and Activity languages. These operators were used to specify three coordination pattern between these languages. We defined the operator \emph{syncFSMEventsAndSignals} that specifies a coordination between FSMEvents and Signals by relying on their names. We defined this operator by slightly modifying the running example operator. Then, we specified the operator \emph{StartActivityWhenEnter} that specifies a coordination between the entering and leaving of a state and the execution of an activity. Finally, we defined the operator \emph{AtomicActivity} to specify how the time is coordinated between these languages.
	
We have noted that some of these coordination patterns are common in coordination frameworks, \eg Ptolemy, ModHel'X. However, their specification is encoded inside a tool and expressed in a GPL, thus limiting the customization of the coordination and the validation of the coordinated system. Conversely, in our approach, operators are explicitly defined by using \bcool, which eases the customization of the operators.
	
We have used these operators to coordinate the heterogeneous models of a surveillance camera system. To do so, we used \bflow to specify how the operators are applied between the different models. Then, in the modeling workbench, we used the \bflow specification to execute the coordinated system.  
	
	
During this thesis, we have defined more operators that can be found in the companion website. For example, we developed the \emph{SyncronizedProduct} operator between the TFSM languages. This is an operator that coordinates TFSM models by synchronizing FSMEvents. We also developed an operator between the TFSM and the SigPML~\cite{moccmlbib} (Signal Processing Modeling Language) languages. SigPML is an extension of SDF in which an application is described as a set of Agents. Upon activation, each agent uses the data on its Input Ports, executes N processing cycles and produces computed results on its Output Ports. In this context, we used \bcool to define an operator that synchronizes the occurrences of FSMEvents and the starting of \emph{Agents} by relying on their names. 

In a more recent work~\cite{combemaleerts16bib}, we have investigated the use of \bcool into \emph{Capella}~\footnote{https://www.polarsys.org/capella/}. The Capella modeling workbench is an Eclipse application implementing the ARCADIA~\footnote{https://www.polarsys.org/capella/arcadia.html} method providing both a DSML and a toolset which is dedicated to guidance, productivity and quality. The Capella DSML aggregates a set of 20 metamodels and about 400 meta-classes involved in the five engineering phases (aka. Architecture level) defined by ARCADIA. In this context, \bcool has been used to define an operator named \emph{ModeEnteringActivateFunctionalChain} that coordinates the action of entering and leaving a mode with the activation of a functionalChain.
	
In the next chapter, we summarize the most important contributions of this thesis and we propose various perspective paths that could be a guide to continue this work. 
	
