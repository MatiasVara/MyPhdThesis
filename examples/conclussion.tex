\section{Conclusion}
In this chapter, we have validated the use of \bcool by capturing a set of hierarchical coordination patterns between the TFSM and Activity languages. These patterns are common in coordination frameworks, \eg Ptolemy, ModHel'X. However, their specification is encoded inside a tool and expressed in a GPL, thus limiting verification and validation activities. To become their specification explicit, we have proposed a set of \bcool operators. We specified the operator \emph{startActivityWhenEnter} that coordinates the entering and leaving of a state and the execution of the activities. Then, we specified the operator \emph{AtomicActivity} to specify how the time is coordinated between these languages. To define the different event relations used in the coordination rule, we based on \moccml. Then, we have used these operators to coordinate the heterogeneous model of a surveillance camera system. To do so, we used \bflow to specify which operators apply on which models. In the modeling workbench, we used the \bflow specification to execute the coordinated system of the camera. In the next chapter, we summarize the most important contributions of this thesis and we propose various perspective paths that could be a guide to continue this work. 

 
