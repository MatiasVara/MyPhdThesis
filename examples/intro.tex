\section{Introduction}
In this chapter, we validate our approach by defining different coordination patterns between the TFSM and Activity languages, which were introduced in Chapter~\ref{ch:bcool}. To motivate and explain the patters, we rely on the coordination of the models of a surveillance camera system. The system is composed of a \emph{Camera Control Encoder} and a \emph{Battery Sensor}. We model the system by using the TFSM and Activity languages, which results in heterogeneous models that need to be coordinated. 
	
To coordinate these models, we propose in this chapter the definition of three \bcool coordination operators between the TFSM and Activity languages. These operators are used to capture three coordination patterns between these languages. One operator synchronizes FSMEvents and Signals by relying on their names. A second operator specifies a hierarchical coordination in which the entering and leaving of states is synchronized with the execution of activities. In addition, we propose a third operator that specifies a timing coordination. More precisely, we specify that the execution of an activity is atomic from point of view of the TFSM language. Thus, during the execution an activity, the time in the TFSM does not elapse. In this chapter, we show how to use \bcool to specify and customize these coordination patterns. Also, we show how the resulting coordination can be used for analysis.
		
We organize this chapter as follows. We begin by presenting the heterogeneous models of a surveillance camera system. To get the global system behavior, we propose to coordinate these models by using three operators in \bcool. Then, we use these operators to coordinate the models, and to execute the coordinated system by using the GEMOC studio. Finally, we conclude.
		

