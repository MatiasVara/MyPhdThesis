\section{Introduction}
In this chapter, we validate our approach by using as use case the coordination of the heterogeneous model of a surveillance camera system. To coordinate these models, we propose to rely on a hierarchical coordination pattern between TFSM and Activity languages. This pattern is common in hierarchical coordination frameworks. However, in frameworks, the semantics of such hierarchical coordination is hidden inside a tool by using a GPL like Java. This limits the customization of the coordination and the verification of the coordinated system.  

We propose to make the specification of this pattern explicit by using two \bcool operators between the TFSM and Activity languages. One captures a hierarchical coordination in which the entering of states is synchronized with the execution of activities. In addition, we propose a second operator that specifies a timing hierarchical coordination between these languages. Then, we use this specification to coordinate and verify the heterogeneous model of a surveillance camera system. 

We organize this chapter as follows. We begin by presenting the heterogeneous model of a surveillance camera system and we propose a set of coordination patterns to coordinate the models. Then, we capture the specification of such coordination patterns by using \bcool operators. We want to highlight that such specification at language level can be used to coordinate any set of TFSMs and Activities models. In particular, in this chapter, we use these operators to coordinate the model of a surveillance video system. Then, we use the GEMOC studio to validate the coordinated model. To finish this chapter, we compare our approach with hierarchical coordination frameworks, and finally we conclude.  