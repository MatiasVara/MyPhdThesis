\section{Introduction}
In this chapter, we validate our approach by using as use case the coordination of the models of a surveillance camera system. The system is composed of a \emph{Camera Control Encoder} and a \emph{Battery Sensor}. We model the system by using the TFSM and Activity languages, which results in a heterogeneous model that need to be coordinated. 


To coordinate these models, we propose in this chapter the definition of three \bcool coordination operators between these languages. One operator synchronizes FSMEvents and Signals by relying on their names. A second operator specifies a hierarchical coordination in which the entering and leaving of states is synchronized with the execution of activities. In addition, we propose a third operator that specifies a timing coordination between these languages. More precisely, we specify that the execution of the activities is atomic from point of view of the TFSM language. Thus, during the execution of the activities, the time in a TFSM does not elapse. In this chapter, we discuss about the different choices in the semantics of the operators. We use such a discussion to compare the customization facilities provided by current coordination frameworks.  

We organize this chapter as follows. We begin by presenting the heterogeneous model of a surveillance camera system. To get the global system behavior, we propose to coordinate these models by using three operators in \bcool. Then, we use these operators to coordinate the models, and we execute the coordinated system by using the GEMOC studio. Finally, we conclude.