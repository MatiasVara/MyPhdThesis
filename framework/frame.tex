\section{The Framework}
\begin{itemize}
	\item We focus on Coordination Pattern approaches. We propose a framework in order to understand better how these approaches are built. More precisely, how these approaches have achieved to capture a coordination pattern between languages. The main objective of the framework is to understand the requirement of a language to capture coordination patterns. We elaborate on such requirements to present \bcool. 


	\item We define a \emph{coordination pattern} as a systematic way to coordinate the behavior of models, based on the knowledge of a system designer. A similar notion of coordination pattern is presented by Hardebolle et al.~\cite{hardebollethese} under the wording of \emph{interaction pattern}. 
	\item An interaction pattern is the formalization of the know-how and the systematic way to solve a coordination problem on heterogeneous languages. The concept of coordination pattern is also present in BIP under the word of \emph{coordination scheme}. It contains all the possible interactions between a set of ports, \eg \emph{Rendezvous}.
	
	\item We define a \emph{coordination pattern approach} an approach that captures a coordination pattern thus leveraging on the know-how a system designer. 
	\item The main characteristic of a coordination pattern approach is they specify a systematic way to coordinate the behavior of models that conform to a set (possibly unlimited) of languages. For instance, MASCOT is limited to Matlab and SDL. 
	  
	\item However, each approach is implemented by using different technologies. In this context, we identified three elements, further described in the next sub sections:  
		
		\item A \emph{language interface}, which provides a (partial) view over the syntax and the behavioral semantics of a language. It is a way, that vary from one approach to other, to homogenize the view of different languages.
		
		\item A \emph{correspondence rule}, which specifies, based on elements from the language interfaces, what elements from different models have to be coordinated together. The notion of correspondence is well defined in~\cite{clavreulmodelcompo}: \emph{we consider correspondences as any kind of implicit or explicit relationships between sets of models or sets of model elements.} 
	
		\item A \emph{coordination rule}, which specifies how a set of formal parameters are coordinated. In the case of model composition, this is defined as a \emph{merging} of elements. By relying on the classification proposed in~\cite{clavreulmodelcompo}, the coordination rule is a particular case of \emph{interpretation}. From authors, \emph{Interpretation is what we define as the meaning of the correspondence relationships for a given purpose in a specific context}. Since the final goal of the coordination rule is to coordinate behavioral models, the interpretation is classified as \emph{interaction}.  
 
\end{itemize}