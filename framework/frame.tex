\section{The Framework}
\subsection{Overview}
\begin{itemize}
	\item We focus on Coordination Pattern approaches. We propose a framework in order to understand better how these approaches are built. More precisely, how these approaches have achieved to capture a coordination pattern between languages. The main objective of the framework is to understand the requirement of a language to capture coordination patterns. We elaborate on such requirements to present \bcool. 

	\item We are inspired by the framework for composition structural approaches presented in~\cite{framecompoas,kompose}: \emph{what?}, \emph{when?} and \emph{how?}.
	
	\item The \emph{what?} and \emph{when?} is related with a phase of \emph{matching} of syntactic model elements.
	
	\item The notion of correspondence is well defined in~\cite{clavreulmodelcompo}: \emph{we consider correspondences as any kind of implicit or explicit relationships between sets of models or sets of model elements.} 
	
	\item \todo{to check the merging in~\cite{clavreulmodelcompo} }
	
	\item The \emph{how?} is related with a phase of merging of syntactic model elements. 
	
	\item A matching rule specifies what elements from different models are selected. A merging rule specifies how the selected model elements are composed. In these approaches the specification is at the language level, but the application is between models.
	
	\item In this context, we identified the following three elements to capture the specification of a coordination pattern between languages: 
		
	\item A \emph{language interface}, which provides a (partial) view over the syntax and the behavioral semantics of a language. It is a way, that vary from one approach to other, to homogenize the view of different languages.
		
	\item A \emph{correspondence rule}, which specifies, based on elements from the language interfaces, what elements from different models have to be coordinated together.
	
	\item A \emph{coordination rule}, which specifies how a set of formal parameters are coordinated. In the case of model composition, this is defined as a \emph{merging} of elements. By relying on the classification proposed in~\cite{clavreulmodelcompo}, the coordination rule is a particular case of \emph{interpretation}. From authors, \emph{Interpretation is what we define as the meaning of the correspondence relationships for a given purpose in a specific context}. Since the final goal of the coordination rule is to coordinate behavioral models, the interpretation is classified as \emph{interaction}.  
 
\end{itemize}