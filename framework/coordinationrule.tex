\section{Coordination Rule}
\subsection{Overview}
The coordination rule defines how the model elements which are in a same correspondence are coordinated together. In other words, it is a specification of the interaction between elements selected by the correspondence rules.

\paragraph{Di Natale et al \& MASCOT :}
in both~\cite{dinatale} and ~\cite{mascotbib}, the coordination rule is specified by the translational semantics of the correspondences, \ie resulting in a C++ glue for Di Natale et al. and in C wrappers for MASCOT. 

These approaches provide a good level of technical adaptation and a low overhead in terms of computational time. However, the drawbacks are strong in terms of flexibility and openness of the approach. First, it requires that all the models and the glue can be specified in a unique language. 
Second, it can be arbitrarily complex to add a third language in these kind of approaches due to the ``intrusive'' nature of the glue, which require an high level of expertise about how the languages are compiled. Finally, the coordination rule is hidden in the compiler, making it opaque to the user and difficult to tune without altering the framework.

%For example, in~\cite{dinatale}, each correspondence in the mapping model (\ie stereotypes) is translated to a specific glue in C++. The glue encodes how elements of the functional and the platform translation semantics are coordinated. More precisely, the glue is encoded into the translation semantics of the mapping language as method invocation in C++. Similarly, MASCOT~\cite{mascotbib} encodes the coordination rule into the SDL wrappers in C. The SDL engine communicates to the Matlab engine by invoking the methods in the SDL wrappers. Such a protocol is implemented in C. 

\paragraph{Ptolemy \& ModHel'X :}			
In both Ptolemy~\cite{ptoleframebib} and ModHel'X~\cite{modhelxbib}, composite actors interact with their internal actors by invoking the methods in their interface. The coordination rules are then expressed in Java together with the semantics of the actors. In ModHel'X, they make explicit a notion of semantic adaptation but only at the model level. It enables the specification of the code that implements the semantic adaptation between two component interfaces. Such a manual adaptation is also done in Java.

These approaches, an particularly Ptolemy, were pioneer in the domain and were very influencing to many current researches. The main drawback we can point about Ptolemy is the distance between the formal works of their authors in the theoretical field of Model of Computations (see \cite{TaggedSignal} for instance) and the implementation of only few of these concepts in Ptolemy. It disables automatic reasoning on the properties of the coordinated system. ModHel'X, mainly copied the concepts of Ptolemy but made explicit the notion of semantic adaptation between two components. They never reified this semantic adaptation at the language level, making the approach close to the notion of glue in existing ADLs.

\subsection{Requirements}

The different approaches reviewed presented some major drawbacks. In this section we enumerate some requirements to help improving these approaches (or any new approaches). 

\cite{dinatale} and~\cite{mascotbib} proposed a coordination rule whose expression is strongly linked to the languages to be coordinated. To allow incorporating new languages into a framework it seems primordial to be able to express the coordination rule independently of the languages whose models must be coordinated. note that this requirement is strongly link with the need for an explicit language behavioral interface.

In the proposed approach the coordination rules are mainly encoded in the framework and not always modifiable by a system designer. To be able to fit particular domains, it may be possible to modify the coordination rules without altering the framework. This is somehow possible in Ptolemy and ModHel'X but since the semantics of a composite actor and the coordination of its internal actors are mixed, there is a risk of altering the semantics of the actor. In ModHel'X, the semantic adaptation can be modified but only at the model level.

None of the approaches relies on a formal language to express the coordination rules. The benefits of having a coordination expressed formally have been highlighted in the ADL community~\cite{wrightbib,rapidebib}. This could be very beneficial to rely on a formal language to express the coordination rules. This, of course, is even more beneficial if the language behavioral interface provides an abstract and formal view of the language behavioral semantics.

To sum up these ideas, we propose four requirements to make existing or new approach more flexible compared to the reviewed ones:

\begin{enumerate}
\item the specification of the coordination rule may be explicit (for understanding) and modifiable (to fit dedicated domains)
\item using a formal language for the coordination rule could be beneficial for further analysis
\end{enumerate}

%In the four reviewed approaches, we identified two similarities as follows:
%	
%\begin{enumerate}
%\item The coordination rule is hidden inside a tool and expressed in a general purpose language;
%\item The customization of the coordination rule is limited.
%\end{enumerate}
%		
%We want to highlight that the reviewed approaches do not leverage on the state-of-art of Coordination Languages and ADLs. For instance, ADLs have addressed 1) by relying on connectors that make explicit the interaction between components. Furthermore, some ADLs express the glue in a formal language to provide verification and validation of the coordinated system. For example, in a recent work~\cite{varagemoc13bib}, authors based on the approach presented in~\cite{sle13-combemale} to coordinate the behavior of heterogeneous models. The coordination is expressed by specifying constraints in \ccsl between the \mse of the model behavioral interface. Then, by using \ccsl tools, authors provided the execution of the coordinated system. In this approach, the coordination is manually specified between two particular models. However, it illustrated the use of \ccsl to express the coordination.  
%		
%Regarding to second point, ADLs have well identified the notion of user-defined connector-types that enables a system designer to build domain specific connectors and use them as needed. In the light of these findings, a coordination rule should be explicitly defined by using a formal language.
%     
	
	
	
	
	
	
	
	
		
 	%\item These are built-in connectors types like AADL or Clara, but using a general purpose language.
	 	%\item \todo{To add the merging operator}
	 	
 		%\item \todo{A recent work~\cite{semanticadaptlang} propose a dedicated language to explicitly specify the semantic adaptation. Currently in ModHel'X, the semantic adaptation is written in java, \ie a general purpose language. Instead, this approach proposes to use a language closed to system designer domain. From DSL code, Java code is generated and implemented into ModHel'X. Since Java code is used, verification and validation of the coordinated system remain limited.}   
 		 		
 		
 		%\item None of the approaches we reviewed really took advantage of state of the art on software architecture and coordination languages. 
 		
 		%\item The notion of user defined connector types as defined by Wright or Reo or Unicon is better formulated the approaches previously presented. Leveraging on such approach to provide coordination rule seems quite intuitive and would bring a clear separation between the computational and coordination activity while allowing the definition of domain specific coordination. 
 		
 		%\item Also, according to the domain addressed by the covered approaches that none of them used a formal language to specify the coordination rule.
 		
 		%\item  Based on state of the art approaches based on formal languages (like BIP or Wright), it can be interesting to experiment the adaptation of a formal language to specify the coordination rule of behavioral patterns between heterogeneous languages.
	
 		%\item A \emph{coordination rule}, which specifies how a set of formal parameters are coordinated. In the case of model composition, this is defined as a \emph{merging} of elements. By relying on the classification proposed in~\cite{clavreulmodelcompo}, the coordination rule is a particular case of \emph{interpretation}. From authors, \emph{Interpretation is what we define as the meaning of the correspondence relationships for a given purpose in a specific context}. Since the final goal of the coordination rule is to coordinate behavioral models, the interpretation is classified as \emph{interaction}.