\section{Coordination Rules}
In this section, we first a review the notion of \emph{Coordination Rules} in existing approaches, and then, we present some requirements to make them better defined. 

\subsection{Review of Existing Approaches}
In the specification of a coordination pattern, coordination rules specify \emph{how} the elements selected by the correspondence rules must be coordinated. Based on a coordination rule, approaches specify the interaction between elements in a correspondence. In the following, we review how existing approaches implement a coordination rule depending on the specified coordination pattern.  

\paragraph{Ptolemy~\cite{ptoleframebib}/ModHel'X~\cite{modhelxbib}:}	
In these approaches, composite actors interact with their internal actors by invoking the methods in their interface. The coordination rule implements a hierarchical execution of actors in which the coordination is expressed in Java together with the semantics of the actors. In ModHel'X, authors made explicit the notion of semantics adaption between actors by allowing the specification of the code between two component interfaces. Such a manual adaptation is also done in Java.

These approaches, in particular Ptolemy, were the first to propose a hierarchical framework to coordinate heterogeneous models. These approaches, however, do not leverage on the formal works done by their authors in the field of Model of Computation~\cite{lee1998framework}. The current implementation contains only few of the ideas proposed in such works. Thus, the main \emph{cons} is that the approaches are limited in terms of verification and validation of the coordinated system. 

ModHel'X went one step forward by proposing the notion of semantics adaptation between two components. However, the semantics adaptation is never reifed at language level thus making the approach close to the notion of glue in existing ADLs.

\paragraph{Di Natale et al.~\cite{dinatale} \& Mascot~\cite{mascotbib}:}
In these approaches, the coordination rule is specified by the translational semantics of the correspondences. In Di Natale et al.~\cite{dinatale}, each correspondence in the mapping model is translated to a specific glue in C++. The glue encodes how elements of the functional and the platform translation semantics are coordinated. In the case of Mascot~\cite{mascotbib}, the coordination rule is encoded by using SDL wrappers which results in a coordination expressed in C.

The \emph{pros} of these approaches is that they achieved to coordinate models that are developed in different technologies, \ie Maltab and SDL. They do so by expressing the semantics of both the coordination and the models in a common GPL, \ie C, C++. 

The main \emph{cons} of these approaches is that they are very limited in terms of customization and explicitness of the coordination rule. They encode the coordination rule into tools thus limiting reasoning about how a particular system is coordinated. In addition, the coordination rule depends on the implementation of the coordinated languages. For instance, in Di Natale et al., the translational semantics of the mapping language depends on the translational semantics of the functional and the platform languages. Thus, the task of adding support to a new language needs a well understanding about the implementation of the coordinated languages.

Another \emph{cons} of these approaches is that they force to express the semantics of both the model and the coordination in the same language. The use of a GPL to express the coordination limits the verification and validation of the resulting coordinated system. 
	
\subsection{Requirements}
In the reviewed approaches, we identified that they encode a different coordination rule depending on the coordination pattern. Such a coordination rule specifies how concepts from different language \emph{must} be coordinated. The coordination rule allows approaches to derivate a glue between the elements selected by a correspondence rule. Such a glue specifies how elements in a correspondence \emph{are} coordinated. However, in these approaches, the coordination rule is encoded into a framework/tool and the coordination is expressed in a GPL. This limits the task of a system designer to provide analysis and verification of the coordinated system. To support the heterogeneous development of complex systems, a system designer has to understand well how a system is coordinated and he must be able to provide analysis of the coordinate system. These characteristics motivate the following requirements:
	\begin{enumerate}
		\item Coordination rules should be expressed independently of the implementation of the coordinated languages;  
		\item Coordination rules should be customizable;
		\item Coordination between models should be formally defined. 
	\end{enumerate}
	
Let us consider each of these requirements in turn. For an approach to specify coordination patterns, it is important that a new language be easy to add. In this sense, the coordination rule cannot depend on the implementation of coordinated languages. \cite{dinatale} and~\cite{mascotbib} proposed a coordination rule whose expression is strongly linked to the implementation of the coordinated languages, thus making tedious to add a new language without modifying the whole implementation. Note that this requirement is strongly linked with the need for an explicit language behavioral interface.
	
In the proposed approaches the coordination rules are encoded in the framework/tools. Thus, a system designer has to modify the current implementation to change the proposed coordination. To be able to specify different coordination patterns, a system designer should be able to modify the coordination rules without altering the whole implementation. This is somehow possible in Ptolemy and ModHel'X but since the semantics of a composite actor and the coordination of its internal actors are mixed, there is a risk of altering the semantics of the actor. In ModHel'X, the semantic adaptation can be modified but only between two particular models.
	
None of the approaches relies on a formal language to express the coordination. The benefits of having a coordination expressed formally have been highlighted in the ADL community~\cite{wrightbib,rapidebib}. A formal description of the coordination together with a formal description of the semantics of models (at least partial) could provide the verification and validation of the coordinated system. In this context, a language behavioral interface could provide an abstract and formal view of the language behavioral semantics.


%In the four reviewed approaches, we identified two similarities as follows:
%	
%\begin{enumerate}
%\item The coordination rule is hidden inside a tool and expressed in a general purpose language;
%\item The customization of the coordination rule is limited.
%\end{enumerate}
%		
%We want to highlight that the reviewed approaches do not leverage on the state-of-art of Coordination Languages and ADLs. For instance, ADLs have addressed 1) by relying on connectors that make explicit the interaction between components. Furthermore, some ADLs express the glue in a formal language to provide verification and validation of the coordinated system. For example, in a recent work~\cite{varagemoc13bib}, authors based on the approach presented in~\cite{sle13-combemale} to coordinate the behavior of heterogeneous models. The coordination is expressed by specifying constraints in \ccsl between the \mse of the model behavioral interface. Then, by using \ccsl tools, authors provided the execution of the coordinated system. In this approach, the coordination is manually specified between two particular models. However, it illustrated the use of \ccsl to express the coordination.  
%		
%Regarding to second point, ADLs have well identified the notion of user-defined connector-types that enables a system designer to build domain specific connectors and use them as needed. In the light of these findings, a coordination rule should be explicitly defined by using a formal language.
%     
	
	
	
	
	
	
	
	
		
 	%\item These are built-in connectors types like AADL or Clara, but using a general purpose language.
	 	%\item \todo{To add the merging operator}
	 	
 		%\item \todo{A recent work~\cite{semanticadaptlang} propose a dedicated language to explicitly specify the semantic adaptation. Currently in ModHel'X, the semantic adaptation is written in java, \ie a general purpose language. Instead, this approach proposes to use a language closed to system designer domain. From DSL code, Java code is generated and implemented into ModHel'X. Since Java code is used, verification and validation of the coordinated system remain limited.}   
 		 		
 		
 		%\item None of the approaches we reviewed really took advantage of state of the art on software architecture and coordination languages. 
 		
 		%\item The notion of user defined connector types as defined by Wright or Reo or Unicon is better formulated the approaches previously presented. Leveraging on such approach to provide coordination rule seems quite intuitive and would bring a clear separation between the computational and coordination activity while allowing the definition of domain specific coordination. 
 		
 		%\item Also, according to the domain addressed by the covered approaches that none of them used a formal language to specify the coordination rule.
 		
 		%\item  Based on state of the art approaches based on formal languages (like BIP or Wright), it can be interesting to experiment the adaptation of a formal language to specify the coordination rule of behavioral patterns between heterogeneous languages.
	
 		%\item A \emph{coordination rule}, which specifies how a set of formal parameters are coordinated. In the case of model composition, this is defined as a \emph{merging} of elements. By relying on the classification proposed in~\cite{clavreulmodelcompo}, the coordination rule is a particular case of \emph{interpretation}. From authors, \emph{Interpretation is what we define as the meaning of the correspondence relationships for a given purpose in a specific context}. Since the final goal of the coordination rule is to coordinate behavioral models, the interpretation is classified as \emph{interaction}.