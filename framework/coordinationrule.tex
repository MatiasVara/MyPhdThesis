\section{Coordination Rule}
In this section, we present the notion of \emph{Coordination Rule}. We first review the notion of coordination rule in existing approaches, and then, we present some requirements to make them better defined. 

\subsection{Review of Existing Approaches}
\begin{itemize}
	\item The coordination rule defines how the model elements that are selected by a correspondence  must be coordinated.
	
	
	\item In the specification of a coordination pattern, coordination rules specify how the elements selected by the correspondences rules must be coordinated.

	
	\item In the following, we review existing approaches by focusing on how they specify the coordination between elements between heterogeneous models.

\end{itemize}


\paragraph{Di Natale et al \& MASCOT :}
\begin{itemize}
	
	\item In these approaches, the coordination rule is specified by the translational semantics of the correspondences. In~\cite{dinatale}, each correspondence in the mapping model is translated to a specific glue in C++. The glue encodes how elements of the functional and the platform translation semantics are coordinated. In the case of MASCOT~\cite{mascotbib}, the coordination rule is encoded by using SDL wrappers in C.
	
	\item The \emph{pros} of these approaches is that they achieved to coordinate models that are developed by using very different technologies, \ie Maltab and SDL. They do so by translating both the coordination and the models into a common GPL, \ie C, C++.
	  
	\item However, both approaches are very limited in terms of customization and explicitness of the coordination. First, they force to express both the model and the coordination in the same language. Second, the glue depends on how each language is implemented. 
	
	\item Thus, the semantics of the coordination should be changed to add support to a new language. For instance, in Di Natale et al., this means to change the behavioral semantics of the mapping language. Thus, The task of adding support to a new language is complex and involves the a well understanding about the implementation of the language.  
		
	\item In both approaches, the coordination rule is encoded into the tools/frameworks. So that, a system designer has to understand very well the frameworks to know how a particular system is coordinated. In addition, the coordination is expressed in a GPL thus limiting verification and validation of the overall system. 
	
\end{itemize} 
%For example, in~\cite{dinatale}, each correspondence in the mapping model (\ie stereotypes) is translated to a specific glue in C++. The glue encodes how elements of the functional and the platform translation semantics are coordinated. More precisely, the glue is encoded into the translation semantics of the mapping language as method invocation in C++. Similarly, MASCOT~\cite{mascotbib} encodes the coordination rule into the SDL wrappers in C. The SDL engine communicates to the Matlab engine by invoking the methods in the SDL wrappers. Such a protocol is implemented in C. 

\paragraph{Ptolemy \& ModHel'X :}	
\begin{itemize}
	\item In these approaches, composite actors interact with their internal actors by invoking the methods in their interface. The coordination rule implements a hierarchical execution of actors that is expressed in Java together with the semantics of actors. In ModHel'X, authors made explicit the notion of semantics adaption between actors. They enable the specification of the code between two component interfaces. Such a manual adaptation is also done in Java.
	
	\item These approaches, in particular Ptolemy, were the first to propose a hierarchical framework for heterogeneous models. These approaches, however, do not leverage on the formal works done by their authors in the field of Model of Computation~\cite{lee1998framework}. The current implementation contains only few of the ideas proposed in such works. Thus, the main \emph{cons} is that the approaches are limited in terms of verification and validation of the coordinated system. 
		
	\item ModHel'X went one step forward by proposing the notion of semantics adaptation between two components. However, the semantics adaptation is never reifed at language level thus making the approach close to the notion of glue in existing ADLs.
	
\end{itemize}		



\subsection{Requirements}

\begin{itemize}
	\item In the reviewed approaches, we identified that each approach encodes a different coordination rule depending on the coordination pattern specified. Such a coordination rule specifies how concepts from different language \emph{must} be coordinated. The coordination rule allows approaches to derivate a glue between model elements. The glue specifies how elements from different models \emph{are} coordinated.   

	\item However, in these approaches, the coordination rule is encoded into a framework/tool and expressed in a GPL. This limits the task of a system designer to provide analysis and verification of the coordinated system. 
	
	\item To support the heterogeneous development of complex systems, a system designer has to understand well how a system is coordinated and he must be able to provide analysis of the coordinate system. These characteristics motivate the following requirements:
	
	\begin{enumerate}
		\item The coordination rule should be independent of the coordinated languages. 
		\item The coordination rule should be customizable. 
		\item The coordination should be formal defined. 
	\end{enumerate}
	
	\item Let us consider each of these requirements in turn. For an approach to specify coordination patterns, it is important that a new language to be easy to add. In this sense, the coordination rule cannot strongly depend on the coordinated languages. \cite{dinatale} and~\cite{mascotbib} proposed a coordination rule whose expression is strongly linked to the languages to be coordinated, thus making tedious to add a new language without modifying the whole implementation. Note that this requirement is strongly link with the need for an explicit language behavioral interface.
	
	\item In the proposed approaches the coordination rules are encoded in the framework/tools. Thus, a system designer has to modify the current implementation to change the proposed coordination. To be able to specify different coordination patterns, a system designer should be able to modify the coordination rules without altering the whole implementation. This is somehow possible in Ptolemy and ModHel'X but since the semantics of a composite actor and the coordination of its internal actors are mixed, there is a risk of altering the semantics of the actor. In ModHel'X, the semantic adaptation can be modified but only between two particular models.
	
	\item None of the approaches relies on a formal language to express the coordination. The benefits of having a coordination expressed formally have been highlighted in the ADL community~\cite{wrightbib,rapidebib}. This could be very beneficial to rely on a formal language to express the coordination. This, of course, is even more beneficial if the language behavioral interface provides an abstract and formal view of the language behavioral semantics.
\end{itemize}

%In the four reviewed approaches, we identified two similarities as follows:
%	
%\begin{enumerate}
%\item The coordination rule is hidden inside a tool and expressed in a general purpose language;
%\item The customization of the coordination rule is limited.
%\end{enumerate}
%		
%We want to highlight that the reviewed approaches do not leverage on the state-of-art of Coordination Languages and ADLs. For instance, ADLs have addressed 1) by relying on connectors that make explicit the interaction between components. Furthermore, some ADLs express the glue in a formal language to provide verification and validation of the coordinated system. For example, in a recent work~\cite{varagemoc13bib}, authors based on the approach presented in~\cite{sle13-combemale} to coordinate the behavior of heterogeneous models. The coordination is expressed by specifying constraints in \ccsl between the \mse of the model behavioral interface. Then, by using \ccsl tools, authors provided the execution of the coordinated system. In this approach, the coordination is manually specified between two particular models. However, it illustrated the use of \ccsl to express the coordination.  
%		
%Regarding to second point, ADLs have well identified the notion of user-defined connector-types that enables a system designer to build domain specific connectors and use them as needed. In the light of these findings, a coordination rule should be explicitly defined by using a formal language.
%     
	
	
	
	
	
	
	
	
		
 	%\item These are built-in connectors types like AADL or Clara, but using a general purpose language.
	 	%\item \todo{To add the merging operator}
	 	
 		%\item \todo{A recent work~\cite{semanticadaptlang} propose a dedicated language to explicitly specify the semantic adaptation. Currently in ModHel'X, the semantic adaptation is written in java, \ie a general purpose language. Instead, this approach proposes to use a language closed to system designer domain. From DSL code, Java code is generated and implemented into ModHel'X. Since Java code is used, verification and validation of the coordinated system remain limited.}   
 		 		
 		
 		%\item None of the approaches we reviewed really took advantage of state of the art on software architecture and coordination languages. 
 		
 		%\item The notion of user defined connector types as defined by Wright or Reo or Unicon is better formulated the approaches previously presented. Leveraging on such approach to provide coordination rule seems quite intuitive and would bring a clear separation between the computational and coordination activity while allowing the definition of domain specific coordination. 
 		
 		%\item Also, according to the domain addressed by the covered approaches that none of them used a formal language to specify the coordination rule.
 		
 		%\item  Based on state of the art approaches based on formal languages (like BIP or Wright), it can be interesting to experiment the adaptation of a formal language to specify the coordination rule of behavioral patterns between heterogeneous languages.
	
 		%\item A \emph{coordination rule}, which specifies how a set of formal parameters are coordinated. In the case of model composition, this is defined as a \emph{merging} of elements. By relying on the classification proposed in~\cite{clavreulmodelcompo}, the coordination rule is a particular case of \emph{interpretation}. From authors, \emph{Interpretation is what we define as the meaning of the correspondence relationships for a given purpose in a specific context}. Since the final goal of the coordination rule is to coordinate behavioral models, the interpretation is classified as \emph{interaction}.