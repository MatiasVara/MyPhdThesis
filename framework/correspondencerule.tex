\section{Correspondence Rules}
This section presents the notion of \emph{Correspondence Rules}. We first review the concept of correspondence rules in existing approaches, and then, we present requirements to make them explicit and better customizable.  

\subsection{Review of Existing Approaches}
A correspondence is any explicit or implicit relationship between model elements. Such a relationship is specified by a system designer that knows what elements from different models are related. In the specification of coordination patterns, correspondences specify the model elements that are coordinated. To automate the process of looking for correspondences between models, the specification of a coordination pattern involves the definition of a correspondence rule at language level. Such a rule defines, in intention, the correspondences to be instantiated at model level. In the following, we review the existing approaches by focusing on how they implemented correspondence rules and correspondences.       
%A correspondence rule is a specification in intention, at the language level, of the correspondences to be created at the model level. In the following we review how the different approaches deal with correspondences and correspondence rules.

\paragraph{Ptolemy~\cite{ptoleframebib}/ModHel'X~\cite{modhelxbib}:}
In these approaches, the notion of composite actors is used to specified the correspondences between models: when an actor is composite (\ie it contains other actors) then it coordinates its internal actors. This enables the designing of a system by following a hierarchical scheme where the level $n$ in the hierarchy specifies the coordination of the models that are at the level $n-1$. These approaches propose a fixed correspondence rule, \ie the composite actor relation, which is encoded into the syntax of the framework.

The \emph{pro} of these approaches is the simplicity for a designer to express the correspondences. He has only to specify what models are composited and which ones are contained inside. The correspondence rule is unique and provided into the common abstract syntax. 

The main \emph{cons} of these approaches is they rely on a unique abstract syntax to describe both the syntax of models and the syntax of the correspondences. This prevents the independent development of the models (possibly developed in different languages) and the correspondences. In addition, the use of a hierarchical design may limit the number of the supported languages. For instance, the UML sequence diagram could not be easy to introduce in the hierarchical view of the framework.

\paragraph{Di Natale et al.~\cite{dinatale}:}
In this work, a mapping model is used to define the correspondences between the functional and the platform model. The mapping model is defined by using a dedicated language (\ie mapping language) that fixes what type of concepts between the platform and the functional languages can be bound together. For instance, the \emph{SwDeployment} correspondence maps two types of concepts: one of type \emph{CPU} and another one of type \emph{Task}. In this sense, the approach provides a set of fixed correspondence rules. 

The \emph{pro} of the approach is that the mapping model defines explicitly the correspondences between the functional and the platform model. Such a model could be useful both reasoning and for traceability. 

The \emph{cons} of this approach is that a system designer has to manually build the mapping model depending on the current deployment. The approach has successfully identified some relationships between a functional and a platform language, which are captured by a set of predefined correspondence rules. From such a correspondence rules, the approach, however, is not able to automate the instantiation of correspondences at model level. 

{\paragraph{Mascot~\cite{mascotbib}}:} In this work, the correspondences are implicit in the models. The approach relies on a naming convention to specify when events in SDL and streams in Matlab must be coordinated. This approach relies on a correspondence rule that specifies that the elements to be coordinated must have the same name. When this rule is applied between models, the framework can automatically get correspondences between model elements.

The \emph{pro} of this approach is the use of a correspondence rule to select model elements. This makes the approach very flexible in terms of dependencies between the languages, thus easing the support for a new language.  

The \emph{cons} is the encoding of the correspondence rule in the framework. Thus, the approach is limited to find correspondences by comparing the name of the elements (\ie events from SDL and streams from Matlab). In some cases, it could be interesting to compare more than the names, \eg the types. In addition, the approach forces the use of a naming convention in the models.     


\subsection{Requirements}
In the reviewed approaches, we identified different correspondence rules to get correspondences between models elements. Ptolemy and ModHel'X rely on a common abstract syntax to identify the elements to be coordinated; Di Natale et al. propose a dedicated language to express an explicit mapping between the elements of different models; and Mascot defines some rules that allow the inference of the correspondences for specific models. However, the main drawback of these approaches is that the customization of the correspondence rules is not possible. Such a characteristic motivates the following requirements for correspondence rules and correspondences:

\begin{enumerate}
	\item Correspondence rules should provide support for heterogeneous languages; 
	\item Correspondence rules should avoid creating dependencies between a predefined set of languages to enable the support of new languages;
	\item Correspondence rules must be explicitly defined and customizable depending on the domain and conventions followed by different system engineers;
	\item Correspondences between model elements should be explicitly represented.  
\end{enumerate}

We discuss each requirement in turn. The current development of complex systems is tackled by relying on several heterogeneous languages where each language has its own syntax and behavioral semantics. Thus, an approach that capture the specification of a coordination pattern must be able to identify correspondences between models that are heterogeneous in terms of the syntax. This prevents the possibility to rely on a common syntax for the languages and the correspondences like in Ptolemy and ModHel'X.


The support of heterogeneous languages should be done in a way that a new language can be easy to add. For instance, in the case of Di Natale et al., the mapping language (\ie the language used for the correspondences) depends on the coordinated languages, (\ie the functional and the platform languages). So that, if any of these languages are modified or a new language has to be added, the mapping language must be changed. Instead, correspondence rules could be defined by relying on partial information exposed in the interface. This would avoid strong dependencies between the correspondence rules and the coordinated languages.

%We believe that the correct specification and use of a homogeneous language interface could also act positively towards this requirement. 

%The first important requirement to face the globalization of modeling languages is the support for heterogeneous languages, not only in terms of semantics but also in terms of syntax. 

%Such support disable the possibility to rely on an artifact of a common abstract syntax like in Ptolemy and ModHel'X. However, it paves the road of an open world with emerging (domain specific) languages as depicted in~\cite{globalization_ieee}.

%Also, in order to address correspondences while taking into account the globalization of modeling languages, it is important to avoid the creation of strong dependencies when specifying the correspondences and the correspondence rules. 

%While this requirements ease the support of new languages, it makes difficult to strongly type or constraint the correspondences. For instance, in \cite{dinatale} the mapping language constrains the mapping model to relate only some specific elements (creating a mapping language with dependencies to the related languages). We believe that the correct specification and use of an homogeneous language interface could also act positively towards this requirement. 

%In Mascot, authors identified that by relying on correspondence rules the process for looking for similarities between models elements can be easily automatized. %These rules are specified at language level and enable to get correspondences between two particular models. However, in MASCOT, these rules are encoded and not well defined. %To improve the definition of correspondence rules, our fourth requirement is that correspondences rules must be explicitly defined and customizable depending on the domain and conventions followed by different system engineers.
A correspondence rule can be used to find correspondences between models that conform to two particular languages. This is the case in~\cite{kofmanbib} where authors use design space exploration techniques to determine the (best) correspondences between an application and its deployment platform. Also, correspondence rules can be used to capture conventions followed by system engineers in an organization. The complexity of the conventions can vary from a simple naming convention (\eg Mascot) to a more complex ontology based system. Based on such conventions, correspondence rules can determine what elements from different models must be coordinated.    

\begin{lstlisting}[language=epsilon, caption={The Mascot correspondence rule in the Epsilon Comparison Language}, label={lst:epsilon}, 	basicstyle=\scriptsize\ttfamily, backgroundcolor=\color{LGrey}, numbers=left, xleftmargin=2pt]
rule MatchEventWithStream
match s : SDLMetamodel!Event
with t : MatlabMetamodel!Stream 
{
compare {
	 return s.name = t.name;
	}
}
\end{lstlisting}

To make correspondence rules explicit, a dedicated language should be used. To illustrate this, we propose to use the Epsilon Comparison Language (ECL~\cite{epsilonbib}) to sketch the definition of the correspondence rule in the case of Mascot. The ECL is a language dedicated to the expression of correspondence rules. In this language, correspondence rules are expressed by querying and then comparing model elements. For example, Listing~\ref{lst:epsilon} shows the specification in ECL of a matching rule named \emph{MatchEventWithStream}. Roughly speaking, the matching rule is used to find correspondences between instances of the \emph{Event} and \emph{Stream} classes (Listing~\ref{lst:epsilon}: line 2 and 3). These classes can be defined in different metamodels (\ie Ecore models). The class Event is defined in an metamodel referred as \emph{SDLMetamodel} and the class Stream is defined in an metamodel referred as \emph{MatlabMetamodel} (Listing~\ref{lst:epsilon}: line 2 and 3). The comparison is done by using the attribute name defined in the context of each class (Listing~\ref{lst:epsilon}: line 5). To express the comparison, the approach relies on a query language named Epsilon Object Language\footnote{http://www.eclipse.org/epsilon/doc/eol/}. When two instances of these classes have the same name, the pairs are matched.

The main benefit of the use of such a dedicated language is to ease the customization of the correspondence rules, thus enabling to adapt them according to the company modeling conventions and the language's nature. For instance in Listing~\ref{lst:epsilon}, the matching is equivalent to the one encoded in Mascot, however, it can be customized to add another criteria for the matching. Furthermore, conventions are independent of the languages themselves, so it is easy to add support for a new language without changing the framework.


Finally, to understand how a particular system is coordinated, correspondences between model elements must be clearly represented. This has already identified in the Ptolemy and ModHel'X by providing a dedicated syntax to specify the correspondences. Unlike these approaches, in Mascot, the correspondences are implicit into the approach thus making necessary to read the documentation to find out that the approach relies on a correspondence rule that follows a naming convention. In such a case, an explicit representation of the correspondences can help a system designer to understand the coordinated elements. Also, it allows external tools to take advantages of the correspondences. The explicit representation of correspondences should be done without creating dependencies between the languages used in the system in order to fulfill the second requirement.% \todo{To fulfill this requirement, interfaces (both at language and model level) could be helpful.}
	
	%\item When a system designer tries to understand the correspondence that exists in a coordinated system, it is very useful for her/him to have an explicit representation of these correspondences.
	
	%\item  It avoids ambiguities in the understanding of the coordination but also it allows external tools to take benefits of the correspondences.
	
	%\item This is for instance not the case in the Mascot approach where the correspondences are never reified and only known by the framework and the people aware of the correspondence rule.
	
	%\item While it brings benefits, it can be challenging to make an explicit correspondence model between heterogeneous models without dependencies to the different languages used in the system.
	
	%\item  Once again, language and model interfaces can be very helpful here. 



	

In this section, we have presented some requirements to improve the way that approaches implements correspondence rules at and correspondences. % We proposed four requirements for a flexible implementation of correspondences and correspondences rules. They should: 
%\begin{enumerate}
%\item provide support for heterogeneous languages;
%\item avoid creating dependencies between a predefined set of languages to enable the support of new languages;
%\item explicitly represented to ease the identification of what are the correspondences between model elements and to understand the rational of such correspondences \ie explicit correspondences rules;
%\item be easy customizable, possibly by using a dedicated language. %adaptable to specific organizations or domains to avoid over constraining the agreements different organizations whose models must be coordinated.
%\end{enumerate}
In the next section, we present the notion of coordination rule that specifies how the models elements selected by a correspondence must be coordinated.     
	
	
%how the notion of correspondences and correspondence rules are dealt with in different approaches focusing on the specification of coordination patterns. Such an artifact is used to identifies when instances of concepts from different languages must be coordinated. 
					
%The first case do not capture all the knowledge of the system designer, which is still injected manually by using a dedicated modeling artifact. It makes lots of sense when the identification of the correspondences is not systematic, like for instance for the allocation of tasks on CPUs. In some cases, this modeling artifact can be created by another dedicated tool (like \cite{kofman:hal-00950533} that uses design space exploration techniques to determine the allocation). It could be interesting to understand how the notions used in the megamodel field~\cite{megamodel} or in multi-view approaches~\cite{ieee42010} can be exploited to specify, in a generic way, the correspondences required in the definition of behavioral coordination patterns between heterogeneous languages.	


%not here !
%Returning to the approach presented in~\cite{sle13-combemale}, the \dse are defined in the context of a metaclass. This can be used to select instances of such \dse (\mse) by querying its context. Furthermore, instances of \dse from different model behavioral interfaces can be selected by comparing elements defined in its context, \ie by using the attribute name. Such a correspondence rule could be implemented by relying on a language to express queries between model elements, \eg OCL, Epsilon Object Language.

	
	%\item It is important to specify some correspondence rules at the language level so that the correspondences can be inferred. 
	
	
	
	
	
	%\item The first considers specific languages on which it can define an analysis to decide what correspondences must be create for some specific models. 
	
				