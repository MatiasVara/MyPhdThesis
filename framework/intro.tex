\section{Introduction}
In this chapter, we present a deep study of approaches that have captured the specification of a coordination pattern. These approaches go one step forward than Coordination Languages and ADLs by leveraging on the know-how of a system designer. They have automated the model coordination by specifying the coordination between languages. However, current implementations seem heterogeneous and not well characterized. This chapter proposes a framework for approaches that have achieved to capture coordination pattern, \ie Ptolemy~\cite{ptoleframebib}, ModHel'X~\cite{modhelxbib}, MASCOT~\cite{mascotbib}, Di Natale et al.~\cite{dinatale}. We use this framework to discuss similarities and differences of these approaches in the light of work done in ADLs and Coordination Languages. 

In the literature, authors proposed frameworks for ADLs~\cite{frameadlsbib}, Coordination Languages~\cite{coordmodels} and Model Composition approaches~\cite{framecompoas}. For example, in~\cite{frameadlsbib}, authors achieved to identify the building-blocks of a ADL. Thus, they proposed a framework in order to classify and compare several existing ADLs. A similar work is presented in~\cite{coordmodels} where authors surveyed several coordination languages. They focus on 1) the entities being coordinated, 2) the mechanism of
coordination, 3) the coordination architecture, 4) the semantics, rules of protocols of coordination employed. In~\cite{framecompoas}, Jeanneret et al. proposed a framework to compare model
composition approaches by relying on the triplet \emph{what-where-how} questions that identify respectively which elements should be composed, where elements should be inserted or modified and how the composition process works to get the expected result. 

By inspired on these works, this chapter presents a framework for language coordination approaches that is based on three questions: \emph{What?}, \emph{When?} and \emph{How?}. These questions identify respectively which languages elements can be coordinated, when elements must be coordinated and how they must be coordinated. The What? is related with a \emph{Language Behavioral Interface} that exposes a partial view of the syntax and behavioral semantics of languages. The When? is related with a \emph{Correspondence Rule} that identifies elements from the interface that must be coordinated. The How? is related with a \emph{Coordination Rule} that defines the interactions between selected elements. 

The goal of the framework is to understand how approaches have achieved to capture a coordination pattern between languages. This chapter is organized as follows. We begin by presenting the notion of language behavioral interface that contains partial information about the syntax and behavioral semantics of the languages that approaches coordinate, then we continue by presenting correspondences between language elements encoded into the notion of correspondence rule, and finally, we present the notion of coordination rule that specifies how the elements selected by the correspondence rule are coordinated. The proposed framework will help us to understand the requirement for a language to explicitly capture coordination patterns. By leveraging on these works, we present a particular implementation of the framework named \bcool.   

