\section{Introduction}
\todo{This INTRO has to be REDO!}
\begin{itemize}
\item In this chapter, we present a deep study on coordination pattern approaches. 

\item These approaches are implemented by using different technologies. However, they can be seen as different implementations (sometimes partial) of a same conceptual framework.

\item The conceptual framework we propose comes from a deep understanding of the different approaches and the internal mechanisms they used. We tried to abstract these mechanisms to extract the essence of the information needed to specify a behavioral coordination pattern between heterogeneous languages. The conceptual framework can be used to better understand the existing approaches but also to define requirements about a more general and systematic way to define such approach.


\item We are inspired by the framework presented in~\cite{framecompoas} to compare model
composition techniques. The framework relies on the triplet \emph{what?}, \emph{where?} and \emph{how?} questions that identify respectively which elements should be composed, where elements should be inserted or modified and how the composition process works to get the expected result. 


\item By relying on this work, we propose a framework for coordination pattern approaches. The goal of this framework is to characterize the mechanisms involved into the coordination pattern approaches. We rely on three questions: \emph{What?}, \emph{When?} and \emph{How?} that identify what concepts can be coordinated, when such elements must be coordination and how the selected elements must be coordinated.   
 

\item In the following, we present these elements.     
	
\end{itemize}
