\section{Introduction}
This chapter proposes a framework to compare and understand the approaches that have achieved to capture coordination patterns, \ie Ptolemy~\cite{ptoleframebib}, ModHel'X~\cite{modhelxbib}, MASCOT~\cite{mascotbib}, Di Natale et al.~\cite{dinatale}. These approaches go one step forward than Coordination Languages and ADLs by leveraging on the know-how of a system designer. They have automated the model coordination by specifying the coordination between languages. However, current implementations seem different and not well characterized. Thus, the objective of the framework is to highlight similarities and differences between these approaches, and to express the requirements to make them more flexible/well founded.

The proposed framework is based on existing comparison/categorization frameworks. In the literature, authors proposed frameworks that focus on ADLs~\cite{frameadlsbib}, Coordination Languages~\cite{coordmodels} and Model Composition approaches~\cite{framecompoas}. For example, in~\cite{frameadlsbib}, authors achieved to identify the primary blocks used to build an ADL. Based on these \emph{building-blocks}, they proposed a framework in order to classify and compare several existing ADLs. A similar work is presented in~\cite{coordmodels} where authors surveyed and classified several coordination languages. They focused on 1) the nature of the entities being coordinated, 2) the coordination mechanism, 3) the coordination architecture and 4) the semantics of the coordination. In~\cite{framecompoas}, Jeanneret et al. proposed a framework to compare model composition approaches by relying on the triplet \emph{what-where-how} questions, which identifies respectively which elements should be composed, where elements should be inserted or modified and how the composition process works to get the expected result. 

Inspired by these works, we propose a framework to characterize any approach that specifies a coordination pattern. Exiting approaches have in common that:
\begin{enumerate}
	\item They specify the coordination between languages;
	\item They specify correspondences between elements from different languages that must be coordinated;
	\item They specify how the selected elements must be coordinated.  
\end{enumerate}	
Regarding to first point, existing approaches specified the coordination at language level by relying on information (at least partial) about the behavioral semantics of the coordinated languages. Therefore, the framework identifies such a representation of the language behavioral semantics as a \emph{Language Behavioral Interface}. Concerning with second and third point, the framework defines respectively a \emph{Correspondence Rule} and a \emph{Coordination Rule}. The former identifies how approaches select elements from different languages to be coordinated, and the latter identifies how approaches specifies the coordination between the selected elements.
%Then, approaches define what and when elements from the interface must be coordinated. Thus, the framework identifies what elements from the interfaces must be coordinated by using the notion of \emph{Correspondence Rule}. Finally, the framework identifies how the selected elements are coordinated by using the notion of \emph{Coordination Rule}.
%Inspired by these works, this chapter presents a framework to compare existing approaches that used coordination patterns specified at the language level. As introduced in section \ref{ch:background}, the coordination is the specification of the interaction, during the execution, of different (pieces of) software. Consequently, in order to specify a coordination pattern, it is important that a language expose (at least partially) its behavioral semantics. This is the aim of the first piece of our framework named \emph{Language Behavioral Interface}. Based on such interfaces, the second piece of our framework is a \emph{Correspondence Rule}, which specifies what elements from the interfaces must be coordinated. Finally, the third piece of our framework is a \emph{Coordination Rule}, which defines how the selected elements from the interface are coordinated.

These three elements made our framework for language coordination approaches. In this chapter, we review how existing approaches implement each element of the framework. By doing so, we highlight the lacks of these approaches to provide a more flexible implementation. Based on such discussion, we propose some requirements for the specification of any coordination pattern by making each element of the framework explicit and customized. 

We organize this chapter as follows. We begin by presenting the notion of language behavioral interface, then we continue by presenting correspondences between elements from the interfaces encoded into the notion of correspondence rule, and finally, we present the notion of coordination rule. Based on the requirements presented in this chapter, Chapter~\ref{ch:bcool} presents a particular implementation of this framework named \bcool.

