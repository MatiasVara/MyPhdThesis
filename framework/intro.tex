\section{Introduction}
In this chapter, we present a deep study of coordination pattern approaches. These approaches go one step forward than Coordination Languages and ADLs by leveraging on the know-how of a system designer. They have automated the model coordination by specifying the coordination between languages. However, current implementation are very heterogeneous and not well characterized. Furthermore, they lack of a relationship with the work done by software architecture and coordination languages. In this chapter, we study the commonalities between these approaches. We do so by proposing a framework for coordination pattern approaches. 

In the literature, authors proposed frameworks for Coordination Languages~\cite{coordmodels}, ADLs~\cite{frameadlsbib}, and Composition of Model Approaches~\cite{framecompoas}. For example, in~\cite{frameadlsbib}, authors propose a framework in order to classify and compare several existing ADLs. A similar work is presented in~\cite{coordmodels} where authors survey several coordination languages. They focus on 1) the entities being coordinated, 2) the mechanism of
coordination, 3) the coordination architecture, 4) the semantics, rules of protocols of coordination employed. In~\cite{framecompoas}, Jeanneret et al. propose a framework to compare model
composition approaches by relying on the triplet \emph{what-where-how} questions that identify respectively which elements should be composed, where elements should be inserted or modified and how the composition process works to get the expected result. 

By inspired on these works, this chapter presents a framework for coordination pattern approaches, which is based on three questions: \emph{What?}, \emph{When?} and \emph{How?}. These questions identify respectively which elements can be coordinated, when elements must be coordinated and how they must coordinated. The What? is related with a \emph{Language Behavioral Interface} that exposes a partial view of the syntax and behavioral semantics of languages. The When? is related with a \emph{Correspondence Rule} that identifies elements from the interface that must be coordinated. The How? is related with a \emph{Coordination Rule} that defines a synchronization/coordination between selected elements. 

The goal of the framework is to understand how approaches have achieved to capture a coordination pattern between languages. The proposed framework will help us to understand the requirement for a language to explicitly capture coordination patterns. We thus conclude this chapter by proposing \bcool.  


