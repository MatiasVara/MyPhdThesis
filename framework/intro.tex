\section{Introduction}
In this chapter, we present a comparative study of approaches that have captured the specification of a coordination pattern. These approaches go one step forward than Coordination Languages and ADLs by leveraging on the know-how of a system designer. They have automated the model coordination by specifying the coordination between languages. The current implementations seem quite different and not well characterized. This chapter proposes a framework to compare and understand the approaches that achieved to capture coordination patterns, \ie Ptolemy~\cite{ptoleframebib}, ModHel'X~\cite{modhelxbib}, MASCOT~\cite{mascotbib}, Di Natale et al.~\cite{dinatale}. This framework first aims at emphasis the similarities and differences of these approaches and second it aims at expressing requirements on existing approaches that would make them more flexible/well founded.

This framework is proposed in the light of work done in ADLs and Coordination Languages. It is also based on existing comparison/categorization frameworks from the literature. It is mainly influenced by three frameworks. One focusing on ADLs~\cite{frameadlsbib}, one focusing on Coordination Languages~\cite{coordmodels} and one focusing on Model Composition approaches~\cite{framecompoas}. For example, in~\cite{frameadlsbib}, authors achieved to identify the primary blocks used to build an ADL. Based on these \emph{building-blocks}, they proposed a framework in order to classify and compare several existing ADLs. A similar work is presented in~\cite{coordmodels} where authors surveyed and classified several coordination languages. They focused on 1) the nature of the entities being coordinated, 2) the coordination mechanism, 3) the coordination architecture and 4) {\color{red}the semantics, rules of protocols of coordination employed}\footnote{\todo{}{\color{red}I'm not sure to understand this piece of sentence, please rephrase or come to talk together}}. In~\cite{framecompoas}, Jeanneret et al. proposed a framework to compare model composition approaches by relying on the triplet \emph{what-where-how} questions, which identifies respectively which elements should be composed, where elements should be inserted or modified and how the composition process works to get the expected result. 


\info{here I was a litte puzzled with the what-when-how stuff. The correspondence rule does not define the \emph{when} ! It is actually part of the what for me (what elements from the LBI must be used). The \emph{When} is more related with the filtering of the occurrences to find the good one to coordinate, \ie to get the good moment in time when the coordination must hold. So I will remove the triplet questions from the newt paragraph. It is only commented in the latex... up to you to revert my changes if you disagree with the previous statements...}

Inspired by these works, this chapter presents a framework to compare existing approaches that used coordination patterns specified at the language level. As introduced in section \ref{ch:background}, the coordination is the specification of the interaction, during the execution, of different (pieces of) software. Consequently, in order to specify a coordination pattern, it is important that a language expose (at least partially) its behavioral semantics. This is the aim of the first piece of our framework named \emph{Language Behavioral Interface}. Based on such interfaces, the second piece of our framework is a \emph{Correspondence Rule}, which specifies what elements from the interfaces must be coordinated. Finally, the third piece of our framework is a \emph{Coordination Rule}, which defines how the selected elements from the interface are coordinated.

%We consequently  of It is based on three questions: \emph{What?}, \emph{When?} and \emph{How?}. These questions identify respectively which languages elements can be coordinated, when elements must be coordinated and how they must be coordinated. The What? is related with a \emph{Language Behavioral Interface} that exposes a partial view of the syntax and behavioral semantics of languages. The When? is related with a \emph{Correspondence Rule} that identifies elements from the interface that must be coordinated. The How? is related with a \emph{Coordination Rule} that defines the interactions between selected elements. 

In this chapter we are explaining how the existing approaches actually fits the proposed framework and we highlight the lacks of existing approaches to provide a flexible approach where the different pieces of the framework are made explicit. Based on this description, we formulate some requirements that any coordination patterns may fulfill.

This chapter is organized as follows. We begin by presenting the notion of language behavioral interface
%that contains partial information about the syntax and behavioral semantics of the coordinated languages
, then we continue by presenting correspondences between elements from the interfaces encoded into the notion of correspondence rule, and finally, we present the notion of coordination rule that specifies how the elements selected by the correspondence rule are coordinated. 
%The proposed framework will help us to understand the requirement for a language to explicitly capture coordination patterns. 
Note that section \ref{ch:bcool} presents a particular implementation of this framework named \bcool. \bcool leverages on the requirements expressed for this framework.

