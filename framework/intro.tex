\section{Introduction}
Language Coordination approaches are used to capture a coordination pattern between languages. These approaches go one step beyond Coordination Languages and ADLs by leveraging on the know-how of a system designer. The coordination scheme is described on the (heterogeneous languages) and then automated for derived models. However, current implementations seem different and not always well characterized. The lack of a systematic way to specify a coordination pattern makes these approaches ad-hoc and not flexible. Furthermore, this prevents a wider adoption of this sort of approaches. In this context, this chapter proposes a framework to compare and understand the approaches that offer solutions to capture coordination patterns. The objective of the framework is to highlight similarities and differences between these approaches, and to express the requirements to make them more flexible and generic.
This is important to be able to address other languages, or other ways of composing them. The goal is not to make a further detailed analysis of state-of-the-art approaches but rather to highlight the good points that we intend to keep in our proposition. 

The proposed framework is mainly influenced by three existing comparison/categorization frameworks. In the literature, authors proposed frameworks that focus on ADLs~\cite{frameadlsbib}, Coordination Languages~\cite{coordmodels} and Model Composition approaches~\cite{framecompoas}. For example, in~\cite{frameadlsbib}, authors identify the primary blocks used to build an ADL. Based on these \emph{building-blocks}, they have proposed a framework to classify and compare several existing ADLs. A similar work is presented in~\cite{coordmodels} where authors surveyed and classified several coordination languages. They focused on 1) the nature of the entities being coordinated, 2) the coordination mechanism, 3) the coordination architecture and 4) the semantics of the coordination. In~\cite{framecompoas}, Jeanneret et al. proposed a framework to compare model composition approaches by relying on the triplet \emph{what-where-how} questions, which identifies respectively which elements should be composed, where elements should be inserted or modified and how the composition process works to get the expected result. 

Inspired by these works, we propose a framework to characterize approaches that specify a coordination pattern. Existing approaches have in common that:
\begin{enumerate}
	\item They specify the coordination between languages;
	
	\item They specify correspondences between elements of the conformed models;
	
	\item They specify how the selected elements must be coordinated.
\end{enumerate}	
Regarding the first point, existing approaches specified the coordination at the language level by relying on information (at least partially) about the behavioral semantics of the coordinated languages. Therefore, the framework identifies such a representation of the language behavioral semantics as a \emph{Language Behavioral Interface}. Concerning the second and third points, the framework identifies respectively \emph{Correspondence Rules} and \emph{Coordination Rules}. The former identify how approaches specify \emph{correspondences} between elements from different models, i.e., the rules that defines which elements of the two models should be coordinated.
The latter identifies how approaches specify the \emph{coordination} between the selected elements, i.e., how the selected elements should be coordinated.
%Then, approaches define what and when elements from the interface must be coordinated. Thus, the framework identifies what elements from the interfaces must be coordinated by using the notion of \emph{Correspondence Rule}. Finally, the framework identifies how the selected elements are coordinated by using the notion of \emph{Coordination Rule}.
%Inspired by these works, this chapter presents a framework to compare existing approaches that used coordination patterns specified at the language level. As introduced in section \ref{ch:background}, the coordination is the specification of the interaction, during the execution, of different (pieces of) software. Consequently, in order to specify a coordination pattern, it is important that a language expose (at least partially) its behavioral semantics. This is the aim of the first piece of our framework named \emph{Language Behavioral Interface}. Based on such interfaces, the second piece of our framework is a \emph{Correspondence Rule}, which specifies what elements from the interfaces must be coordinated. Finally, the third piece of our framework is a \emph{Coordination Rule}, which defines how the selected elements from the interface are coordinated.

For this first contribution, we review some of the approaches discussed in the previous chapter and we identify precisely what we consider to be related to the language behavioral interface, the correspondence rules and the coordination rules. We also highlight the mechanisms used and what can, in our view, be improved to build a generic and flexible coordination language.
We then propose some requirements to improve existing approaches by making each element of the framework explicit and better customizable.  

We organize this chapter as follows. We begin by presenting the notion of language behavioral interface, then we continue by presenting correspondences between elements from the interfaces encoded into the notion of correspondence rule, and finally, we present the notion of coordination rule. Based on the requirements presented in this chapter, Chapter~\ref{ch:bcool} presents a particular implementation of this framework named \bcool.

